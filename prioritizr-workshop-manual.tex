\documentclass[12pt,]{book}
\usepackage{lmodern}
\usepackage{amssymb,amsmath}
\usepackage{ifxetex,ifluatex}
\usepackage{fixltx2e} % provides \textsubscript
\ifnum 0\ifxetex 1\fi\ifluatex 1\fi=0 % if pdftex
  \usepackage[T1]{fontenc}
  \usepackage[utf8]{inputenc}
\else % if luatex or xelatex
  \ifxetex
    \usepackage{mathspec}
  \else
    \usepackage{fontspec}
  \fi
  \defaultfontfeatures{Ligatures=TeX,Scale=MatchLowercase}
\fi
% use upquote if available, for straight quotes in verbatim environments
\IfFileExists{upquote.sty}{\usepackage{upquote}}{}
% use microtype if available
\IfFileExists{microtype.sty}{%
\usepackage{microtype}
\UseMicrotypeSet[protrusion]{basicmath} % disable protrusion for tt fonts
}{}
\usepackage[left=2.54cm, right=2.54cm, top=2.54cm, bottom=2.54cm]{geometry}
\usepackage{hyperref}
\PassOptionsToPackage{usenames,dvipsnames}{color} % color is loaded by hyperref
\hypersetup{unicode=true,
            pdftitle={PRIORITIZR WORKSHOP MANUAL},
            pdfauthor={Jeffrey O. Hanson (with modifications by Richard Schuster)},
            colorlinks=true,
            linkcolor=Maroon,
            citecolor=Blue,
            urlcolor=blue,
            breaklinks=true}
\urlstyle{same}  % don't use monospace font for urls
\usepackage{natbib}
\bibliographystyle{plainnat}
\usepackage{color}
\usepackage{fancyvrb}
\newcommand{\VerbBar}{|}
\newcommand{\VERB}{\Verb[commandchars=\\\{\}]}
\DefineVerbatimEnvironment{Highlighting}{Verbatim}{commandchars=\\\{\}}
% Add ',fontsize=\small' for more characters per line
\usepackage{framed}
\definecolor{shadecolor}{RGB}{248,248,248}
\newenvironment{Shaded}{\begin{snugshade}}{\end{snugshade}}
\newcommand{\KeywordTok}[1]{\textcolor[rgb]{0.13,0.29,0.53}{\textbf{#1}}}
\newcommand{\DataTypeTok}[1]{\textcolor[rgb]{0.13,0.29,0.53}{#1}}
\newcommand{\DecValTok}[1]{\textcolor[rgb]{0.00,0.00,0.81}{#1}}
\newcommand{\BaseNTok}[1]{\textcolor[rgb]{0.00,0.00,0.81}{#1}}
\newcommand{\FloatTok}[1]{\textcolor[rgb]{0.00,0.00,0.81}{#1}}
\newcommand{\ConstantTok}[1]{\textcolor[rgb]{0.00,0.00,0.00}{#1}}
\newcommand{\CharTok}[1]{\textcolor[rgb]{0.31,0.60,0.02}{#1}}
\newcommand{\SpecialCharTok}[1]{\textcolor[rgb]{0.00,0.00,0.00}{#1}}
\newcommand{\StringTok}[1]{\textcolor[rgb]{0.31,0.60,0.02}{#1}}
\newcommand{\VerbatimStringTok}[1]{\textcolor[rgb]{0.31,0.60,0.02}{#1}}
\newcommand{\SpecialStringTok}[1]{\textcolor[rgb]{0.31,0.60,0.02}{#1}}
\newcommand{\ImportTok}[1]{#1}
\newcommand{\CommentTok}[1]{\textcolor[rgb]{0.56,0.35,0.01}{\textit{#1}}}
\newcommand{\DocumentationTok}[1]{\textcolor[rgb]{0.56,0.35,0.01}{\textbf{\textit{#1}}}}
\newcommand{\AnnotationTok}[1]{\textcolor[rgb]{0.56,0.35,0.01}{\textbf{\textit{#1}}}}
\newcommand{\CommentVarTok}[1]{\textcolor[rgb]{0.56,0.35,0.01}{\textbf{\textit{#1}}}}
\newcommand{\OtherTok}[1]{\textcolor[rgb]{0.56,0.35,0.01}{#1}}
\newcommand{\FunctionTok}[1]{\textcolor[rgb]{0.00,0.00,0.00}{#1}}
\newcommand{\VariableTok}[1]{\textcolor[rgb]{0.00,0.00,0.00}{#1}}
\newcommand{\ControlFlowTok}[1]{\textcolor[rgb]{0.13,0.29,0.53}{\textbf{#1}}}
\newcommand{\OperatorTok}[1]{\textcolor[rgb]{0.81,0.36,0.00}{\textbf{#1}}}
\newcommand{\BuiltInTok}[1]{#1}
\newcommand{\ExtensionTok}[1]{#1}
\newcommand{\PreprocessorTok}[1]{\textcolor[rgb]{0.56,0.35,0.01}{\textit{#1}}}
\newcommand{\AttributeTok}[1]{\textcolor[rgb]{0.77,0.63,0.00}{#1}}
\newcommand{\RegionMarkerTok}[1]{#1}
\newcommand{\InformationTok}[1]{\textcolor[rgb]{0.56,0.35,0.01}{\textbf{\textit{#1}}}}
\newcommand{\WarningTok}[1]{\textcolor[rgb]{0.56,0.35,0.01}{\textbf{\textit{#1}}}}
\newcommand{\AlertTok}[1]{\textcolor[rgb]{0.94,0.16,0.16}{#1}}
\newcommand{\ErrorTok}[1]{\textcolor[rgb]{0.64,0.00,0.00}{\textbf{#1}}}
\newcommand{\NormalTok}[1]{#1}
\usepackage{longtable,booktabs}
\usepackage{graphicx}
% grffile has become a legacy package: https://ctan.org/pkg/grffile
\IfFileExists{grffile.sty}{%
\usepackage{grffile}
}{}
\makeatletter
\def\maxwidth{\ifdim\Gin@nat@width>\linewidth\linewidth\else\Gin@nat@width\fi}
\def\maxheight{\ifdim\Gin@nat@height>\textheight\textheight\else\Gin@nat@height\fi}
\makeatother
% Scale images if necessary, so that they will not overflow the page
% margins by default, and it is still possible to overwrite the defaults
% using explicit options in \includegraphics[width, height, ...]{}
\setkeys{Gin}{width=\maxwidth,height=\maxheight,keepaspectratio}
\IfFileExists{parskip.sty}{%
\usepackage{parskip}
}{% else
\setlength{\parindent}{0pt}
\setlength{\parskip}{6pt plus 2pt minus 1pt}
}
\setlength{\emergencystretch}{3em}  % prevent overfull lines
\providecommand{\tightlist}{%
  \setlength{\itemsep}{0pt}\setlength{\parskip}{0pt}}
\setcounter{secnumdepth}{5}
% Redefines (sub)paragraphs to behave more like sections
\ifx\paragraph\undefined\else
\let\oldparagraph\paragraph
\renewcommand{\paragraph}[1]{\oldparagraph{#1}\mbox{}}
\fi
\ifx\subparagraph\undefined\else
\let\oldsubparagraph\subparagraph
\renewcommand{\subparagraph}[1]{\oldsubparagraph{#1}\mbox{}}
\fi

%%% Use protect on footnotes to avoid problems with footnotes in titles
\let\rmarkdownfootnote\footnote%
\def\footnote{\protect\rmarkdownfootnote}

%%% Change title format to be more compact
\usepackage{titling}

% Create subtitle command for use in maketitle
\providecommand{\subtitle}[1]{
  \posttitle{
    \begin{center}\large#1\end{center}
    }
}

\setlength{\droptitle}{-2em}

  \title{PRIORITIZR WORKSHOP MANUAL}
    \pretitle{\vspace{\droptitle}\centering\huge}
  \posttitle{\par}
    \author{Jeffrey O. Hanson (with modifications by Richard Schuster)}
    \preauthor{\centering\large\emph}
  \postauthor{\par}
      \predate{\centering\large\emph}
  \postdate{\par}
    \date{2019-11-14}

% load packages
\usepackage{caption}
\usepackage{float}

% default bookdown preamble
\usepackage{booktabs}
\usepackage{amsthm}
\makeatletter
\def\thm@space@setup{%
  \thm@preskip=8pt plus 2pt minus 4pt
  \thm@postskip=\thm@preskip
}
\makeatother

% remove figure labelling
\captionsetup[figure]{labelformat=empty,textfont=it}

% make figures static
\let\origfigure\figure
\let\endorigfigure\endfigure
\renewenvironment{figure}[1][2] {
  \expandafter\origfigure\expandafter[H]
} {
  \endorigfigure
}

% text boxes
\ifxetex
  \usepackage{letltxmacro}
  \setlength{\XeTeXLinkMargin}{1pt}
  \LetLtxMacro\SavedIncludeGraphics\includegraphics
  \def\includegraphics#1#{% #1 catches optional stuff (star/opt. arg.)
    \IncludeGraphicsAux{#1}%
  }%
  \newcommand*{\IncludeGraphicsAux}[2]{%
    \XeTeXLinkBox{%
      \SavedIncludeGraphics#1{#2}%
    }%
  }%
\fi

\makeatletter
\newenvironment{kframe}{%
\medskip{}
\setlength{\fboxsep}{.8em}
 \def\at@end@of@kframe{}%
 \ifinner\ifhmode%
  \def\at@end@of@kframe{\end{minipage}}%
  \begin{minipage}{\columnwidth}%
 \fi\fi%
 \def\FrameCommand##1{\hskip\@totalleftmargin \hskip-\fboxsep
 \colorbox{shadecolor}{##1}\hskip-\fboxsep
     % There is no \\@totalrightmargin, so:
     \hskip-\linewidth \hskip-\@totalleftmargin \hskip\columnwidth}%
 \MakeFramed {\advance\hsize-\width
   \@totalleftmargin\z@ \linewidth\hsize
   \@setminipage}}%
 {\par\unskip\endMakeFramed%
 \at@end@of@kframe}
\makeatother

\makeatletter
\@ifundefined{Shaded}{
}{\renewenvironment{Shaded}{\begin{kframe}}{\end{kframe}}}
\makeatother

\newenvironment{rmdblock}[1]
  {
  \begin{itemize}
  \renewcommand{\labelitemi}{
    \raisebox{-.7\height}[0pt][0pt]{
      {\setkeys{Gin}{width=3em,keepaspectratio}\includegraphics{images/#1}}
    }
  }
  \setlength{\fboxsep}{1em}
  \begin{kframe}
  \item
  }
  {
  \end{kframe}
  \end{itemize}
  }
\newenvironment{rmdnote}
  {\begin{rmdblock}{note}}
  {\end{rmdblock}}
\newenvironment{rmdcaution}
  {\begin{rmdblock}{caution}}
  {\end{rmdblock}}
\newenvironment{rmdquestion}
  {\begin{rmdblock}{question}}
  {\end{rmdblock}}
\newenvironment{rmdanswer}
  {\begin{rmdblock}{answer}}
  {\end{rmdblock}}
\newenvironment{rmdimportant}
  {\begin{rmdblock}{important}}
  {\end{rmdblock}}
\newenvironment{rmdtip}
  {\begin{rmdblock}{tip}}
  {\end{rmdblock}}
\newenvironment{rmdwarning}
  {\begin{rmdblock}{warning}}
  {\end{rmdblock}}

\let\BeginKnitrBlock\begin \let\EndKnitrBlock\end
\begin{document}
\maketitle

{
\hypersetup{linkcolor=black}
\setcounter{tocdepth}{0}
\tableofcontents
}
\chapter{Welcome!}\label{welcome}

Here you will find the manual for the prioritizr module of the
\href{https://pacmara.org/}{\emph{Introduction to Marxan \& MarZone \&
Prioritizr - Training Course}} held at University of Victoria, Victoria,
Canada. \textbf{Before you arrive at the workshop, you should make sure
that you have correctly \protect\hyperlink{setup}{set up your computer
for the workshop} and you have
\href{https://github.com/prioritizr/PacMara_workshop/raw/master/data.zip}{downloaded
the data from here}. We cannot guarantee a reliable Internet connection
during the workshop, and so you may be unable to complete the workshop
if you have not set up your computer beforehand.}

\chapter{Introduction}\label{introduction}

\section{Overview}\label{overview}

The aim of this workshop is to help you get started with using the
prioritizr R package for systematic conservation planning. It is not
designed to give you a comprehensive overview and you will not become an
expert after completing this workshop. Instead, we want to help you
understand the core principles of conservation planning and guide you
through some of the common tasks involved with developing
prioritizations. In other words, we want to give you the knowledge base
and confidence needed to start applying systematic conservation planning
to your own work.

You are not alone in this workshop. If you are having trouble, please
put your hand up and one of the instructors will help you as soon as
they can. You can also ask the people sitting next to you for help too.
\textbf{Most importantly, the code needed to answer the questions in
this workshop are almost always located in the same section as the
question. So if you are stuck, try rereading the example code and see if
you can modify it to answer the question.} Please note that the first
thing an instructor will ask you will be ``what have you tried so
far?''. We can't help you if you haven't tried anything.

\hypertarget{setup}{\section{Setting up your computer}\label{setup}}

You will need to have both \href{https://www.r-project.org}{R} and
\href{https://www.rstudio.com/}{RStudio} installed on your computer to
complete this workshop. Although it is not imperative that you have the
latest version of RStudio installed, \textbf{you will need the latest
version of R installed (i.e.~version 3.6.1)}. Please note that you might
need administrative permissions to install these programs. After
installing them, you will also need to install some R packages too.

\subsection{R}\label{r}

The \href{https://www.r-project.org}{R statistical computing
environment} can be downloaded from the Comprehensive R Archive Network
(CRAN). Specifically, you can download the latest version of R (version
3.6.1) from here: \url{https://cloud.r-project.org}. Please note that
you will need to download the correct file for your operating system
(i.e.~Linux, Mac OSX, Windows).

\subsection{RStudio}\label{rstudio}

\href{https://www.rstudio.com}{RStudio} is an integrated development
environment (IDE). In other words, it is a program that is designed to
make your R programming experience more enjoyable. During this workshop,
you will interact with R through RStudio---meaning that you will open
RStudio to code in R. You can download the latest version of RStudio
here: \url{http://www.rstudio.com/download}. When you start RStudio, you
will see two main parts of the interface:

\begin{center}\includegraphics[width=1\linewidth]{images/rstudio-console} \end{center}

You can type R code into the \emph{Console} and press the enter key to
run code.

\subsection{R packages}\label{r-packages}

An R package is a collection of R code and documentation that can be
installed to enhance the standard R environment with additional
functionality. Currently, there are over fifteen thousand R packages
available on CRAN. Each of these R packages are developed to perform a
specific task, such as
\href{https://cran.r-project.org/web/packages/readxl/index.html}{reading
Excel spreadsheets},
\href{https://cran.r-project.org/web/packages/MODIStsp/index.html}{downloading
satellite imagery data},
\href{https://cran.r-project.org/web/packages/wdpar/index.html}{downloading
and cleaning protected area data}, or
\href{https://cran.r-project.org/web/packages/ENMeval/index.html}{fitting
environmental niche models}. In fact, R has such a diverse ecosystem of
R packages, that the question is almost always not ``can I use R to
\ldots{}?'' but ``what R package can I use to \ldots{}?''. During this
workshop, we will use several R packages. To install these R packages,
please enter the code below in the \emph{Console} part of the RStudio
interface and press enter. Note that you will require an Internet
connection and the installation process may take some time to complete.

\begin{Shaded}
\begin{Highlighting}[]
\KeywordTok{install.packages}\NormalTok{(}\KeywordTok{c}\NormalTok{(}\StringTok{"sf"}\NormalTok{, }\StringTok{"tidyverse"}\NormalTok{, }\StringTok{"sp"}\NormalTok{, }\StringTok{"rgeos"}\NormalTok{, }\StringTok{"rgdal"}\NormalTok{, }\StringTok{"raster"}\NormalTok{,}
                   \StringTok{"units"}\NormalTok{, }\StringTok{"prioritizr"}\NormalTok{, }\StringTok{"prioritizrdata"}\NormalTok{, }\StringTok{"Rsymphony"}\NormalTok{,}
                   \StringTok{"mapview"}\NormalTok{, }\StringTok{"assertthat"}\NormalTok{, }\StringTok{"velox"}\NormalTok{, }\StringTok{"remotes"}\NormalTok{,}
                   \StringTok{"gridExtra"}\NormalTok{, }\StringTok{"data.table"}\NormalTok{, }\StringTok{"readxl"}\NormalTok{, }\StringTok{"BiocManager"}\NormalTok{))}
\NormalTok{BiocManager}\OperatorTok{::}\KeywordTok{install}\NormalTok{(}\StringTok{"lpsymphony"}\NormalTok{, }\DataTypeTok{version =} \StringTok{"3.9"}\NormalTok{)}
\end{Highlighting}
\end{Shaded}

\section{Further reading}\label{further-reading}

There is a wealth of resources available for learning how to use R.
Although not required for this workshop, I would highly recommend that
you read \href{https://r4ds.had.co.nz/}{\emph{R for Data Science} by
Garrett Grolemund and Hadley Wickham}. \textbf{This veritable trove of R
goodness is freely available online.} If you spend a week going through
this book then you will save months debugging and rerunning incorrect
code. I would urge any and all ecologists, especially those working on
Masters or PhD degrees, to read this book. I even bought this book as a
Christmas present for my sister---and, yes, she was happy to receive it!
For intermediate users looking to skill-up, I would recommend the
\href{http://shop.oreilly.com/product/9781593273842.do}{\emph{The Art of
R Programming: A Tour of Statistical Software Design} by Norman Matloff}
and \href{https://adv-r.hadley.nz/}{\emph{Advanced R} by Hadley
Wickham}. Finally, if you wish to learn more about using R as a
geospatial information system (GIS), I would recommend
\href{https://geocompr.robinlovelace.net/}{\emph{Geocomputation with R}
by Robin Lovelace, Jakub Nowosad, and Jannes Muenchow} which is also
freely available online. I also recommend
\href{https://www.springer.com/gp/book/9781461476177}{\emph{Applied
Spatial Data Analysis} by Roger S. Bivand, Edzer Pebesma, and Virgilio
Gómez-Rubio} too.

\chapter{Redo Marxan analysis}\label{marxan-analysis}

\section{Introduction}\label{introduction-1}

Before we begin to prioritize areas for protected area establishment
using the full feature set of \emph{prioritizr}, we will re-do the
Marxan analysis from Tuesday in \emph{prioritizr}. This exercise is
meant to show you how you can use your current Marxan files in
prioritizr, if you choose to do so. Once we have run the example using
input.dat, as well as the individual .dat files, we will also work on
preparing the data you worked with on Monday to be used in prioritzr.
Once that's complete, we will run our first \emph{prioritizr} analysis,
using the notation typical for prioritizr analysis.

The data for this exercise were provided by
\href{https://pacmara.org/}{PacMara} and
\href{https://bcmca.ca/maps-data/browse-or-search/}{BCMCA}.

\section{Starting out}\label{starting-out}

We will start by opening RStudio. Ideally, you will have already
installed both R and Rstudio before the workshop. If you have not done
this already, then please see the \protect\hyperlink{setup}{Setting up
your computer} section. \textbf{During this workshop, please do not copy
and paste code from the workshop manual into RStudio. Instead, please
write it out yourself in an R script.} When programming, you will spend
a lot of time fixing coding mistakes---that is, debugging your code---so
it is best to get used to making mistakes now when you have people here
to help you. You can create a new R script by clicking on \emph{File} in
the RStudio menu bar, then \emph{New File}, and then \emph{R Script}.

\begin{center}\includegraphics[width=0.7\linewidth]{images/rstudio-new-script} \end{center}

After creating a new script, you will notice that a new \emph{Source}
panel has appeared. In the \emph{Source} panel, you can type and edit
code before you run it. You can run code in the \emph{Source} panel by
placing the cursor (i.e.~the blinking line) on the desired line of code
and pressing \texttt{Control\ +\ Enter} on your keyboard (or
\texttt{CMD\ +\ Enter} if you are using an Apple computer). You can save
the code in the \emph{Source} panel by pressing \texttt{Control\ +\ s}
on your keyboard (or \texttt{CMD\ +\ s} if you are using an Apple
computer).

\begin{center}\includegraphics[width=0.7\linewidth]{images/rstudio-source} \end{center}

You can also make notes and write your answers to the workshop questions
inside the R script. When writing notes and answers, add a \texttt{\#}
symbol so that the text following the \texttt{\#} symbol is treated as a
comment and not code. This means that you don't have to worry about
highlighting specific parts of the script to avoid errors.

\begin{Shaded}
\begin{Highlighting}[]
\CommentTok{# this is a comment and R will ignore this text if you run it}
\CommentTok{# R will run the code below because it does not start with a # symbol}
\KeywordTok{print}\NormalTok{(}\StringTok{"this is not a comment"}\NormalTok{)}
\end{Highlighting}
\end{Shaded}

\begin{verbatim}
## [1] "this is not a comment"
\end{verbatim}

\begin{Shaded}
\begin{Highlighting}[]
\CommentTok{# you can also add comments to the same line of R code too}
\KeywordTok{print}\NormalTok{(}\StringTok{"this is also not a comment"}\NormalTok{) }\CommentTok{# but this is a comment}
\end{Highlighting}
\end{Shaded}

\begin{verbatim}
## [1] "this is also not a comment"
\end{verbatim}

\textbf{Remember to save your script regularly to ensure that you don't
lose anything in the event that RStudio crashes (e.g.~using
\texttt{Control\ +\ s} or \texttt{CMD\ +\ s})!}

\section{Attaching packages}\label{attaching-packages}

Now we will set up our R session for the workshop. Specifically, enter
the following R code to attach the R packages used in this workshop.

\begin{Shaded}
\begin{Highlighting}[]
\CommentTok{# load packages}
\KeywordTok{library}\NormalTok{(tidyverse)}
\KeywordTok{library}\NormalTok{(prioritizr)}
\KeywordTok{library}\NormalTok{(rgdal)}
\KeywordTok{library}\NormalTok{(raster)}
\KeywordTok{library}\NormalTok{(rgeos)}
\KeywordTok{library}\NormalTok{(mapview)}
\KeywordTok{library}\NormalTok{(units)}
\KeywordTok{library}\NormalTok{(scales)}
\KeywordTok{library}\NormalTok{(assertthat)}
\KeywordTok{library}\NormalTok{(gridExtra)}
\KeywordTok{library}\NormalTok{(data.table)}
\KeywordTok{library}\NormalTok{(readxl)}
\end{Highlighting}
\end{Shaded}

\section{Base analysis on input.dat}\label{base-analysis-on-input.dat}

Now we will redo the Marxan analyis you have done on Tuesday, but using
\emph{prioritzr}. To do so we need the \emph{Marxan database} you used
on Tuesday, as well we the \emph{raw data} you used on Monday. The files
for both are already included in the R Studio project you received for
this exercise. Now please open the \emph{PacMara\_workshop.Rproj} file
by double clicking it. You are now ready to start with the exercise.

First, we are going to use the information from the input.dat file to
run the analysis you completed on Tuesday, using \emph{prioritzr}. To do
so, all you need to do is point to input.dat and tell \emph{prioritizr}
where to find it. Once that's done we can generate the problem and solve
it.

\begin{Shaded}
\begin{Highlighting}[]
\NormalTok{input_file <-}\StringTok{ "Marxan_database/input.dat"}

\NormalTok{p1 <-}\StringTok{ }\KeywordTok{marxan_problem}\NormalTok{(input_file)}

\NormalTok{s1 <-}\StringTok{ }\KeywordTok{solve}\NormalTok{(p1)}
\end{Highlighting}
\end{Shaded}

Next, we are going to have a look at the solution and explore the output
by first displaying a couple of rows from the output data, then counting
the number of planning units in the solution and calcualating the
proportion of planning units in the solution.

\begin{Shaded}
\begin{Highlighting}[]
\KeywordTok{head}\NormalTok{(s1)}
\end{Highlighting}
\end{Shaded}

\begin{verbatim}
##   id    cost status locked_in locked_out solution_1
## 1  1 2000000      0     FALSE      FALSE          1
## 2  2 2000000      0     FALSE      FALSE          1
## 3  3 2000000      0     FALSE      FALSE          1
## 4  4 2000000      0     FALSE      FALSE          1
## 5  5 2000000      0     FALSE      FALSE          1
## 6  6 2000000      0     FALSE      FALSE          1
\end{verbatim}

\begin{Shaded}
\begin{Highlighting}[]
\CommentTok{# count number of planning units in solution}
\KeywordTok{sum}\NormalTok{(s1}\OperatorTok{$}\NormalTok{solution_}\DecValTok{1}\NormalTok{)}
\end{Highlighting}
\end{Shaded}

\begin{verbatim}
## [1] 3262
\end{verbatim}

\begin{Shaded}
\begin{Highlighting}[]
\CommentTok{# proportion of planning units in solution}
\KeywordTok{mean}\NormalTok{(s1}\OperatorTok{$}\NormalTok{solution_}\DecValTok{1}\NormalTok{)}
\end{Highlighting}
\end{Shaded}

\begin{verbatim}
## [1] 0.2683669
\end{verbatim}

Next we are going to explore how well the features are represented in
the solution.

\begin{Shaded}
\begin{Highlighting}[]
\CommentTok{# calculate feature representation}
\NormalTok{r1 <-}\StringTok{ }\KeywordTok{feature_representation}\NormalTok{(p1, s1[, }\StringTok{"solution_1"}\NormalTok{, }\DataTypeTok{drop =} \OtherTok{FALSE}\NormalTok{])}
\KeywordTok{print}\NormalTok{(r1)}
\end{Highlighting}
\end{Shaded}

\begin{verbatim}
## # A tibble: 19 x 3
##    feature      absolute_held relative_held
##    <chr>                <dbl>         <dbl>
##  1 0-20 Hard        213510000         0.300
##  2 0-20 Muddy       145420000         0.302
##  3 0-20 Sandy        45810000         0.301
##  4 20-50 Hard       842380000         0.300
##  5 20-50 Muddy      120580000         0.300
##  6 20-50 Sandy       93400000         0.300
##  7 200+ Hard        136480000         0.300
##  8 200+ Muddy       870960000         0.300
##  9 200+ Sandy      1075960000         0.300
## 10 200+ UnId       2025510000         0.300
## 11 50-200 Hard     2151230000         0.300
## 12 50-200 Muddy    1080500000         0.300
## 13 50-200 Sandy    3347250000         0.300
## 14 50-200 UnId        3260000         0.337
## 15 iba             1991070000         0.300
## 16 kelp              51940000         0.305
## 17 killer whale    1596950000         0.386
## 18 sealions         697060000         0.494
## 19 seaotters       1596000000         0.3
\end{verbatim}

Finally, we are going to visualize the solution by converting the
solution to a spatial object.

\begin{Shaded}
\begin{Highlighting}[]
\NormalTok{pulayer <-}\StringTok{ }\KeywordTok{readOGR}\NormalTok{(}\StringTok{"Marxan_database/pulayer/pulayer_BC_marine.shp"}\NormalTok{, }\DataTypeTok{stringsAsFactors =} \OtherTok{FALSE}\NormalTok{)}
\end{Highlighting}
\end{Shaded}

\begin{verbatim}
## OGR data source with driver: ESRI Shapefile 
## Source: "/home/travis/build/prioritizr/PacMara_workshop/Marxan_database/pulayer/pulayer_BC_marine.shp", layer: "pulayer_BC_marine"
## with 12155 features
## It has 1 fields
## Integer64 fields read as strings:  PUID
\end{verbatim}

\begin{Shaded}
\begin{Highlighting}[]
\NormalTok{pulayer1 <-}\StringTok{ }\NormalTok{pulayer}
\NormalTok{pulayer1}\OperatorTok{$}\NormalTok{solution_}\DecValTok{1}\NormalTok{ <-}\StringTok{ }\NormalTok{s1}\OperatorTok{$}\NormalTok{solution_}\DecValTok{1}
\NormalTok{pulayer1}\OperatorTok{$}\NormalTok{solution_}\DecValTok{1}\NormalTok{ <-}\StringTok{ }\KeywordTok{factor}\NormalTok{(pulayer1}\OperatorTok{$}\NormalTok{solution_}\DecValTok{1}\NormalTok{)}
\KeywordTok{spplot}\NormalTok{(pulayer1, }\StringTok{"solution_1"}\NormalTok{, }\DataTypeTok{col.regions =} \KeywordTok{c}\NormalTok{(}\StringTok{"grey90"}\NormalTok{, }\StringTok{"darkgreen"}\NormalTok{),}
       \DataTypeTok{main =} \StringTok{"marxan_problem solution"}\NormalTok{)}
\end{Highlighting}
\end{Shaded}

\begin{center}\includegraphics{prioritizr-workshop-manual_files/figure-latex/unnamed-chunk-12-1} \end{center}

Now, think about the following questions.

\BeginKnitrBlock{rmdquestion}
\begin{enumerate}
\def\labelenumi{\arabic{enumi}.}
\tightlist
\item
  Are the results from Marxan and prioritizr the same/similar?
\item
  If you see differences, why do you think those differences occur?
\item
  Can you think of ways to reduce difference/improve outcomes?
\end{enumerate}
\EndKnitrBlock{rmdquestion}

\clearpage

\section{Base analysis using individual .dat
files}\label{base-analysis-using-individual-.dat-files}

Now, lets redo the analysis, but instead of using input.dat, we will use
the individual .dat files to create the problem. You will see that the
syntax for the problem formulation is very similar, but instead of
supplying one value to the \emph{marxan\_problem} function, we now
specify \emph{pu}, \emph{spec}, \emph{puvsp} and \emph{bound}. If you
want to learn more about the \emph{marxan\_problem} function, just type
in \texttt{?marxan\_problem} and you can have a look at the function
help page.

\begin{Shaded}
\begin{Highlighting}[]
\NormalTok{pu <-}\StringTok{ }\KeywordTok{fread}\NormalTok{(}\StringTok{"Marxan_database/input/pu.dat"}\NormalTok{, }\DataTypeTok{data.table =} \OtherTok{FALSE}\NormalTok{)}
\NormalTok{spec <-}\StringTok{ }\KeywordTok{fread}\NormalTok{(}\StringTok{"Marxan_database/input/spec.dat"}\NormalTok{, }\DataTypeTok{data.table =} \OtherTok{FALSE}\NormalTok{)}
\NormalTok{puvsp <-}\StringTok{ }\KeywordTok{fread}\NormalTok{(}\StringTok{"Marxan_database/input/puvsp.dat"}\NormalTok{, }\DataTypeTok{data.table =} \OtherTok{FALSE}\NormalTok{)}
\NormalTok{bound <-}\StringTok{ }\KeywordTok{fread}\NormalTok{(}\StringTok{"Marxan_database/input/bound.dat"}\NormalTok{, }\DataTypeTok{data.table =} \OtherTok{FALSE}\NormalTok{)}

\NormalTok{p2 <-}\StringTok{ }\KeywordTok{marxan_problem}\NormalTok{(}\DataTypeTok{x =}\NormalTok{ pu, }\DataTypeTok{spec =}\NormalTok{ spec, }\DataTypeTok{puvspr =}\NormalTok{ puvsp)}

\NormalTok{s2 <-}\StringTok{ }\KeywordTok{solve}\NormalTok{(p2)}

\CommentTok{# count number of planning units in solution}
\KeywordTok{sum}\NormalTok{(s2}\OperatorTok{$}\NormalTok{solution_}\DecValTok{1}\NormalTok{)}
\end{Highlighting}
\end{Shaded}

\begin{verbatim}
## [1] 3261
\end{verbatim}

\begin{Shaded}
\begin{Highlighting}[]
\CommentTok{# proportion of planning units in solution}
\KeywordTok{mean}\NormalTok{(s2}\OperatorTok{$}\NormalTok{solution_}\DecValTok{1}\NormalTok{)}
\end{Highlighting}
\end{Shaded}

\begin{verbatim}
## [1] 0.2682847
\end{verbatim}

\begin{Shaded}
\begin{Highlighting}[]
\CommentTok{# calculate feature representation}
\NormalTok{r2 <-}\StringTok{ }\KeywordTok{feature_representation}\NormalTok{(p2, s2[, }\StringTok{"solution_1"}\NormalTok{, }\DataTypeTok{drop =} \OtherTok{FALSE}\NormalTok{])}
\KeywordTok{print}\NormalTok{(r2)}
\end{Highlighting}
\end{Shaded}

\begin{verbatim}
## # A tibble: 19 x 3
##    feature      absolute_held relative_held
##    <chr>                <dbl>         <dbl>
##  1 0-20 Hard        213530000         0.300
##  2 0-20 Muddy       145420000         0.302
##  3 0-20 Sandy        45940000         0.302
##  4 20-50 Hard       842510000         0.300
##  5 20-50 Muddy      120590000         0.300
##  6 20-50 Sandy       93640000         0.301
##  7 200+ Hard        136630000         0.300
##  8 200+ Muddy       871080000         0.300
##  9 200+ Sandy      1075970000         0.300
## 10 200+ UnId       2025510000         0.300
## 11 50-200 Hard     2150800000         0.300
## 12 50-200 Muddy    1081840000         0.301
## 13 50-200 Sandy    3347430000         0.300
## 14 50-200 UnId        3260000         0.337
## 15 iba             1992240000         0.300
## 16 kelp              51570000         0.303
## 17 killer whale    1594570000         0.385
## 18 sealions         702660000         0.498
## 19 seaotters       1596000000         0.3
\end{verbatim}

\section{Recreate the Marxan analysis starting from the raw
data}\label{recreate-the-marxan-analysis-starting-from-the-raw-data}

Now that we have solved a problem that was formatted the way Marxan
needs data, lets go ahead and start from the raw data, as you did on
Monday.

We will first process the data, so we can use it in \emph{prioriztr} and
then we will create the problem and solve it in the `standard'
\emph{prioritizr} way.

Starting with the raw data from folder \emph{Marxan\_Data} we will go
ahead and create our problem. First, lets load all the data we need in
terms of features:

\begin{Shaded}
\begin{Highlighting}[]
\CommentTok{# We first load this Excel file to extract feature names later}
\NormalTok{feat_ids <-}\StringTok{ }\KeywordTok{read_xlsx}\NormalTok{(}\StringTok{"Marxan_Data/conservation_feats_ids.xlsx"}\NormalTok{)}

\CommentTok{# now lets load all the rasters we need}
\NormalTok{iba <-}\StringTok{ }\KeywordTok{raster}\NormalTok{(}\StringTok{"Marxan_Data/iba_bc.tif"}\NormalTok{)}
\NormalTok{kelp <-}\StringTok{ }\KeywordTok{raster}\NormalTok{(}\StringTok{"Marxan_Data/kelp_bc.tif"}\NormalTok{)}
\NormalTok{killerw <-}\StringTok{ }\KeywordTok{raster}\NormalTok{(}\StringTok{"Marxan_Data/killerwhale.tif"}\NormalTok{)}
\NormalTok{seal <-}\StringTok{ }\KeywordTok{raster}\NormalTok{(}\StringTok{"Marxan_Data/sealions.tif"}\NormalTok{)}
\NormalTok{seao <-}\StringTok{ }\KeywordTok{raster}\NormalTok{(}\StringTok{"Marxan_Data/seaotter.tif"}\NormalTok{)}

\NormalTok{benthic <-}\StringTok{ }\KeywordTok{raster}\NormalTok{(}\StringTok{"Marxan_Data/benthic14cl.tif"}\NormalTok{)}

\CommentTok{# benthic and the rest of the rasters are not exactly in the same format (same number of rows and columns)}
\CommentTok{# so we need to go ahead and make sure benthic has the same format as the other rasters.}
\NormalTok{benthic <-}\StringTok{ }\KeywordTok{resample}\NormalTok{(benthic, iba, }\DataTypeTok{method=}\StringTok{"ngb"}\NormalTok{)}
\end{Highlighting}
\end{Shaded}

Now that we have loaded all the feature data into R, we need to go ahead
and create the benthic classes, you have used to setup the Marxan
problem before. This is specific to the way the benthic raster is setup
and will differ from case to case in real world examples you might
explore in the future.

In this specfic case, we know that benthic has a total of 14 classes, so
the R code below does split the benthic raster up into 14 rasters, based
on cell values, and at the end puts them together in a stack.

\begin{Shaded}
\begin{Highlighting}[]
\NormalTok{benthic_values <-}\StringTok{ }\KeywordTok{values}\NormalTok{(benthic)}
\NormalTok{ben_list <-}\StringTok{ }\KeywordTok{list}\NormalTok{()}
\ControlFlowTok{for}\NormalTok{(ii }\ControlFlowTok{in} \DecValTok{1}\OperatorTok{:}\DecValTok{14}\NormalTok{)\{}
\NormalTok{  tmp_r <-}\StringTok{ }\NormalTok{benthic}
\NormalTok{  tmp_r_val <-}\StringTok{ }\NormalTok{benthic_values}
\NormalTok{  tmp_r_val <-}\StringTok{ }\KeywordTok{ifelse}\NormalTok{(tmp_r_val }\OperatorTok{==}\StringTok{ }\NormalTok{ii, }\DecValTok{1}\NormalTok{, }\OtherTok{NA}\NormalTok{)}
  \KeywordTok{values}\NormalTok{(tmp_r) <-}\StringTok{ }\NormalTok{tmp_r_val}
  
\NormalTok{  ben_list[[ii]] <-}\StringTok{ }\NormalTok{tmp_r}
    
  \KeywordTok{rm}\NormalTok{(tmp_r, tmp_r_val)}
  
\NormalTok{\}}
\NormalTok{ben_stack <-}\StringTok{ }\KeywordTok{stack}\NormalTok{(ben_list)}
\end{Highlighting}
\end{Shaded}

Now that all rasters have been created, we can combine them in a stack
and give them the names from the Excel file we read in earlier.

\begin{Shaded}
\begin{Highlighting}[]
\NormalTok{features <-}\StringTok{ }\KeywordTok{stack}\NormalTok{(ben_stack, iba, kelp, killerw, seal, seao)}
\KeywordTok{names}\NormalTok{(features) <-}\StringTok{ }\NormalTok{feat_ids}\OperatorTok{$}\NormalTok{New_Name}
\end{Highlighting}
\end{Shaded}

Next, we load in the fishcost layer and also resample it to fit the rest
of the raster layers.

\begin{Shaded}
\begin{Highlighting}[]
\NormalTok{fishcost <-}\StringTok{ }\KeywordTok{raster}\NormalTok{(}\StringTok{"Marxan_Data/fishcost.tif"}\NormalTok{)}
\NormalTok{fishcost <-}\StringTok{ }\KeywordTok{resample}\NormalTok{(fishcost, iba, }\DataTypeTok{method=}\StringTok{"ngb"}\NormalTok{)}
\end{Highlighting}
\end{Shaded}

Now its time to setup the \emph{prioritizr} problem. As a first step, we
are reading in the pulayer from the \emph{Marxan\_database}. I'm doing
this to show you a nice way to setup the \emph{prioritizr} problem,
using a shapefile directly in the \texttt{problem} function call. We
need to extract the fishcost data to that pulayer, before we can use
this information in the \texttt{problem} formulation.

\begin{Shaded}
\begin{Highlighting}[]
\NormalTok{pulayer <-}\StringTok{ }\KeywordTok{readOGR}\NormalTok{(}\StringTok{"Marxan_database/pulayer/pulayer_BC_marine.shp"}\NormalTok{, }\DataTypeTok{stringsAsFactors =} \OtherTok{FALSE}\NormalTok{)}
\end{Highlighting}
\end{Shaded}

\begin{verbatim}
## OGR data source with driver: ESRI Shapefile 
## Source: "/home/travis/build/prioritizr/PacMara_workshop/Marxan_database/pulayer/pulayer_BC_marine.shp", layer: "pulayer_BC_marine"
## with 12155 features
## It has 1 fields
## Integer64 fields read as strings:  PUID
\end{verbatim}

\begin{Shaded}
\begin{Highlighting}[]
\NormalTok{pulayer}\OperatorTok{$}\NormalTok{cost <-}\StringTok{ }\KeywordTok{as.vector}\NormalTok{(}\KeywordTok{fast_extract}\NormalTok{(fishcost, pulayer))}
\end{Highlighting}
\end{Shaded}

Now for the actual \texttt{problem} formulation. You will see that we
use the pulayer as one of the inputs to the \texttt{problem} function.
As pulayer is a shapefile, we need to tell \emph{prioritzr} which
attribute to use as the cost column. We also include the features raster
stack directly in the \texttt{problem} function. We also set the
objective function (minumum set), the relative targets (0.3 or 30\% of
each feature), and the decision type (binary for integer linear
programming).

When that's done we can solve the problem.

\begin{Shaded}
\begin{Highlighting}[]
\NormalTok{p3 <-}\StringTok{ }\KeywordTok{problem}\NormalTok{(pulayer, }\DataTypeTok{cost_column =} \StringTok{"cost"}\NormalTok{, }\DataTypeTok{features =}\NormalTok{ features, }\DataTypeTok{run_checks =} \OtherTok{FALSE}\NormalTok{) }\OperatorTok
\StringTok{  }\KeywordTok{add_min_set_objective}\NormalTok{() }\OperatorTok
\StringTok{  }\KeywordTok{add_relative_targets}\NormalTok{(}\FloatTok{0.3}\NormalTok{) }\OperatorTok
\StringTok{  }\KeywordTok{add_binary_decisions}\NormalTok{() }

\NormalTok{s3 <-}\StringTok{ }\KeywordTok{solve}\NormalTok{(p3)}
\end{Highlighting}
\end{Shaded}

As we have done before, we will now go ahead and extract summary
statistics as well as plot the results. We just worked through an entire
\emph{prioritzr} problem, from reading and processing raw data to
setting up and solving a \texttt{problem}, to extracting statistics and
spatial visualization of results.

\begin{Shaded}
\begin{Highlighting}[]
\CommentTok{# count number of planning units in solution}
\KeywordTok{sum}\NormalTok{(s3}\OperatorTok{$}\NormalTok{solution_}\DecValTok{1}\NormalTok{, }\DataTypeTok{na.rm =} \OtherTok{TRUE}\NormalTok{)}
\end{Highlighting}
\end{Shaded}

\begin{verbatim}
## [1] 2332
\end{verbatim}

\begin{Shaded}
\begin{Highlighting}[]
\CommentTok{# proportion of planning units in solution}
\KeywordTok{mean}\NormalTok{(s3}\OperatorTok{$}\NormalTok{solution_}\DecValTok{1}\NormalTok{, }\DataTypeTok{na.rm =} \OtherTok{TRUE}\NormalTok{)}
\end{Highlighting}
\end{Shaded}

\begin{verbatim}
## [1] 0.1919816
\end{verbatim}

\begin{Shaded}
\begin{Highlighting}[]
\NormalTok{s3}\OperatorTok{$}\NormalTok{solution_}\DecValTok{1}\NormalTok{ <-}\StringTok{ }\KeywordTok{factor}\NormalTok{(s3}\OperatorTok{$}\NormalTok{solution_}\DecValTok{1}\NormalTok{)}
\KeywordTok{spplot}\NormalTok{(s3, }\StringTok{"solution_1"}\NormalTok{, }\DataTypeTok{col.regions =} \KeywordTok{c}\NormalTok{(}\StringTok{"grey90"}\NormalTok{, }\StringTok{"darkgreen"}\NormalTok{),}
       \DataTypeTok{main =} \StringTok{"problem solution"}\NormalTok{)}
\end{Highlighting}
\end{Shaded}

\begin{center}\includegraphics{prioritizr-workshop-manual_files/figure-latex/unnamed-chunk-21-1} \end{center}

\chapter{Data}\label{data}

You should have already downloaded the data for the prioritizr module of
this workshop. If you have not already done so, you can download it from
here:
\url{https://github.com/prioritizr/PacMara_workshop/raw/master/data.zip}.
After downloading the data, you can unzip the data into a new folder.
Next, you will need to set the working directory to this new folder. To
achieve this, click on the \emph{Session} button on the RStudio menu
bar, then click \emph{Set Working Directory}, and then \emph{Choose
Directory}.

\begin{center}\includegraphics[width=0.7\linewidth]{images/rstudio-wd} \end{center}

\clearpage

Now navigate to the folder where you unzipped the data and select
\emph{Open}. You can verify that you have correctly set the working
directory using the following R code. You should see the output
\texttt{TRUE} in the \emph{Console} panel.

\begin{Shaded}
\begin{Highlighting}[]
\KeywordTok{file.exists}\NormalTok{(}\StringTok{"data/pu.shp"}\NormalTok{)}
\end{Highlighting}
\end{Shaded}

\begin{verbatim}
## [1] TRUE
\end{verbatim}

\section{Data import}\label{data-import}

Now that we have downloaded the dataset, we will need to import it into
our R session. Specifically, this data was obtained from the
``Introduction to Marxan'' course and was originally a subset of a
larger spatial prioritization project performed under contract to
Australia's Department of Environment and Water Resources. It contains
vector-based planning unit data (\texttt{pu.shp}) and the raster-based
data describing the spatial distributions of 62 vegetation classes
(\texttt{vegetation.tif}) in Tasmania, Australia. Please note this
dataset is only provided for teaching purposes and should not be used
for any real-world conservation planning. We can import the data into
our R session using the following code.

\begin{Shaded}
\begin{Highlighting}[]
\CommentTok{# import planning unit data}
\NormalTok{pu_data <-}\StringTok{ }\KeywordTok{readOGR}\NormalTok{(}\StringTok{"data/pu.shp"}\NormalTok{)}
\end{Highlighting}
\end{Shaded}

\begin{verbatim}
## OGR data source with driver: ESRI Shapefile 
## Source: "/home/travis/build/prioritizr/PacMara_workshop/data/pu.shp", layer: "pu"
## with 1130 features
## It has 5 fields
\end{verbatim}

\begin{Shaded}
\begin{Highlighting}[]
\CommentTok{# format columns in planning unit data}
\NormalTok{pu_data}\OperatorTok{$}\NormalTok{locked_in <-}\StringTok{ }\KeywordTok{as.logical}\NormalTok{(pu_data}\OperatorTok{$}\NormalTok{locked_in)}
\NormalTok{pu_data}\OperatorTok{$}\NormalTok{locked_out <-}\StringTok{ }\KeywordTok{as.logical}\NormalTok{(pu_data}\OperatorTok{$}\NormalTok{locked_out)}

\CommentTok{# import vegetation data}
\NormalTok{veg_data <-}\StringTok{ }\KeywordTok{stack}\NormalTok{(}\StringTok{"data/vegetation.tif"}\NormalTok{)}
\end{Highlighting}
\end{Shaded}

\clearpage

\section{Planning unit data}\label{planning-unit-data}

The planning unit data contains spatial data describing the geometry for
each planning unit and attribute data with information about each
planning unit (e.g.~cost values). Let's investigate the
\texttt{pu\_data} object. The attribute data contains 5 columns with
contain the following information:

\begin{itemize}
\tightlist
\item
  \texttt{id}: unique identifiers for each planning unit
\item
  \texttt{cost}: acquisition cost values for each planning unit
  (millions of Australian dollars).
\item
  \texttt{status}: status information for each planning unit (only
  relevant with Marxan)
\item
  \texttt{locked\_in}: logical values (i.e.
  \texttt{TRUE}/\texttt{FALSE}) indicating if planning units are covered
  by protected areas or not.
\item
  \texttt{locked\_out}: logical values (i.e.
  \texttt{TRUE}/\texttt{FALSE}) indicating if planning units cannot be
  managed as a protected area because they contain are too degraded.
\end{itemize}

\begin{Shaded}
\begin{Highlighting}[]
\CommentTok{# print a short summary of the data}
\KeywordTok{print}\NormalTok{(pu_data)}
\end{Highlighting}
\end{Shaded}

\begin{verbatim}
## class       : SpatialPolygonsDataFrame 
## features    : 1130 
## extent      : 1080623, 1399989, -4840595, -4497092  (xmin, xmax, ymin, ymax)
## crs         : +proj=aea +lat_1=-18 +lat_2=-36 +lat_0=0 +lon_0=132 +x_0=0 +y_0=0 +ellps=GRS80 +units=m +no_defs 
## variables   : 5
## names       :   id,              cost, status, locked_in, locked_out 
## min values  :    1, 0.192488262910798,      0,         0,          0 
## max values  : 1130,  61.9272727272727,      2,         1,          1
\end{verbatim}

\begin{Shaded}
\begin{Highlighting}[]
\CommentTok{# plot the planning unit data}
\KeywordTok{plot}\NormalTok{(pu_data)}
\end{Highlighting}
\end{Shaded}

\begin{center}\includegraphics{prioritizr-workshop-manual_files/figure-latex/unnamed-chunk-28-1} \end{center}

\begin{Shaded}
\begin{Highlighting}[]
\CommentTok{# plot an interactive map of the planning unit data}
\KeywordTok{mapview}\NormalTok{(pu_data)}
\end{Highlighting}
\end{Shaded}

\begin{Shaded}
\begin{Highlighting}[]
\CommentTok{# print the structure of object}
\KeywordTok{str}\NormalTok{(pu_data, }\DataTypeTok{max.level =} \DecValTok{2}\NormalTok{)}
\end{Highlighting}
\end{Shaded}

\begin{verbatim}
## Formal class 'SpatialPolygonsDataFrame' [package "sp"] with 5 slots
##   ..@ data       :'data.frame':  1130 obs. of  5 variables:
##   ..@ polygons   :List of 1130
##   ..@ plotOrder  : int [1:1130] 217 973 506 645 705 975 253 271 704 889 ...
##   ..@ bbox       : num [1:2, 1:2] 1080623 -4840595 1399989 -4497092
##   .. ..- attr(*, "dimnames")=List of 2
##   ..@ proj4string:Formal class 'CRS' [package "sp"] with 1 slot
\end{verbatim}

\begin{Shaded}
\begin{Highlighting}[]
\CommentTok{# print the class of the object}
\KeywordTok{class}\NormalTok{(pu_data)}
\end{Highlighting}
\end{Shaded}

\begin{verbatim}
## [1] "SpatialPolygonsDataFrame"
## attr(,"package")
## [1] "sp"
\end{verbatim}

\begin{Shaded}
\begin{Highlighting}[]
\CommentTok{# print the slots of the object}
\KeywordTok{slotNames}\NormalTok{(pu_data)}
\end{Highlighting}
\end{Shaded}

\begin{verbatim}
## [1] "data"        "polygons"    "plotOrder"   "bbox"        "proj4string"
\end{verbatim}

\begin{Shaded}
\begin{Highlighting}[]
\CommentTok{# print the geometry for the 80th planning unit}
\NormalTok{pu_data}\OperatorTok{@}\NormalTok{polygons[[}\DecValTok{80}\NormalTok{]]}
\end{Highlighting}
\end{Shaded}

\begin{verbatim}
## An object of class "Polygons"
## Slot "Polygons":
## [[1]]
## An object of class "Polygon"
## Slot "labpt":
## [1]  1289177 -4558185
## 
## Slot "area":
## [1] 1060361
## 
## Slot "hole":
## [1] FALSE
## 
## Slot "ringDir":
## [1] 1
## 
## Slot "coords":
##          [,1]     [,2]
##  [1,] 1288123 -4558431
##  [2,] 1287877 -4558005
##  [3,] 1288177 -4558019
##  [4,] 1288278 -4558054
##  [5,] 1288834 -4558038
##  [6,] 1289026 -4557929
##  [7,] 1289168 -4557928
##  [8,] 1289350 -4557790
##  [9,] 1289517 -4557744
## [10,] 1289618 -4557773
## [11,] 1289836 -4557965
## [12,] 1290000 -4557984
## [13,] 1290025 -4557987
## [14,] 1290144 -4558168
## [15,] 1290460 -4558431
## [16,] 1288123 -4558431
## 
## 
## 
## Slot "plotOrder":
## [1] 1
## 
## Slot "labpt":
## [1]  1289177 -4558185
## 
## Slot "ID":
## [1] "79"
## 
## Slot "area":
## [1] 1060361
\end{verbatim}

\begin{Shaded}
\begin{Highlighting}[]
\CommentTok{# print the coordinate reference system}
\KeywordTok{print}\NormalTok{(pu_data}\OperatorTok{@}\NormalTok{proj4string)}
\end{Highlighting}
\end{Shaded}

\begin{verbatim}
## CRS arguments:
##  +proj=aea +lat_1=-18 +lat_2=-36 +lat_0=0 +lon_0=132 +x_0=0 +y_0=0
## +ellps=GRS80 +units=m +no_defs
\end{verbatim}

\begin{Shaded}
\begin{Highlighting}[]
\CommentTok{# print number of planning units (geometries) in the data}
\KeywordTok{nrow}\NormalTok{(pu_data)}
\end{Highlighting}
\end{Shaded}

\begin{verbatim}
## [1] 1130
\end{verbatim}

\begin{Shaded}
\begin{Highlighting}[]
\CommentTok{# print the first six rows in the attribute data}
\KeywordTok{head}\NormalTok{(pu_data}\OperatorTok{@}\NormalTok{data)}
\end{Highlighting}
\end{Shaded}

\begin{verbatim}
##   id     cost status locked_in locked_out
## 0  1 60.24638      0     FALSE       TRUE
## 1  2 19.86301      0     FALSE      FALSE
## 2  3 59.68051      0     FALSE       TRUE
## 3  4 32.41614      0     FALSE      FALSE
## 4  5 26.17706      0     FALSE      FALSE
## 5  6 51.26218      0     FALSE       TRUE
\end{verbatim}

\begin{Shaded}
\begin{Highlighting}[]
\CommentTok{# print the first six values in the cost column of the attribute data}
\KeywordTok{head}\NormalTok{(pu_data}\OperatorTok{$}\NormalTok{cost)}
\end{Highlighting}
\end{Shaded}

\begin{verbatim}
## [1] 60.24638 19.86301 59.68051 32.41614 26.17706 51.26218
\end{verbatim}

\begin{Shaded}
\begin{Highlighting}[]
\CommentTok{# print the highest cost value}
\KeywordTok{max}\NormalTok{(pu_data}\OperatorTok{$}\NormalTok{cost)}
\end{Highlighting}
\end{Shaded}

\begin{verbatim}
## [1] 61.92727
\end{verbatim}

\begin{Shaded}
\begin{Highlighting}[]
\CommentTok{# print the smallest cost value}
\KeywordTok{min}\NormalTok{(pu_data}\OperatorTok{$}\NormalTok{cost)}
\end{Highlighting}
\end{Shaded}

\begin{verbatim}
## [1] 0.1924883
\end{verbatim}

\begin{Shaded}
\begin{Highlighting}[]
\CommentTok{# print average cost value}
\KeywordTok{mean}\NormalTok{(pu_data}\OperatorTok{$}\NormalTok{cost)}
\end{Highlighting}
\end{Shaded}

\begin{verbatim}
## [1] 25.13536
\end{verbatim}

\begin{Shaded}
\begin{Highlighting}[]
\CommentTok{# plot a map of the planning unit cost data}
\KeywordTok{spplot}\NormalTok{(pu_data, }\StringTok{"cost"}\NormalTok{)}
\end{Highlighting}
\end{Shaded}

\begin{center}\includegraphics[width=0.6\linewidth]{prioritizr-workshop-manual_files/figure-latex/unnamed-chunk-30-1} \end{center}

\begin{Shaded}
\begin{Highlighting}[]
\CommentTok{# plot an interactive map of the planning unit cost data}
\KeywordTok{mapview}\NormalTok{(pu_data, }\DataTypeTok{zcol =} \StringTok{"cost"}\NormalTok{)}
\end{Highlighting}
\end{Shaded}

Now, you can try and answer some questions about the planning unit data.

\BeginKnitrBlock{rmdquestion}
\begin{enumerate}
\def\labelenumi{\arabic{enumi}.}
\tightlist
\item
  How many planning units are in the planning unit data?
\item
  What is the highest cost value?
\item
  How many planning units are covered by the protected areas (hint:
  \texttt{sum(x)})?
\item
  What is the proportion of the planning units that are covered by the
  protected areas (hint: \texttt{mean(x)})?
\item
  How many planning units are highly degraded (hint: \texttt{sum(x)})?
\item
  What is the proportion of planning units are highly degraded (hint:
  \texttt{mean(x)})?
\item
  Can you verify that all values in the \texttt{locked\_in} and
  \texttt{locked\_out} columns are zero or one (hint: \texttt{min(x)}
  and \texttt{max(x)})?.
\item
  Can you verify that none of the planning units are missing cost values
  (hint: \texttt{all(is.finite(x))})?.
\item
  Can you very that none of the planning units have duplicated
  identifiers? (hint: \texttt{sum(duplicated(x))})?
\item
  Is there a spatial pattern in the planning unit cost values (hint: use
  \texttt{spplot} to make a map).
\item
  Is there a spatial pattern in where most planning units are covered by
  protected areas (hint: use \texttt{spplot} to make a map).
\end{enumerate}
\EndKnitrBlock{rmdquestion}

\clearpage

\section{Vegetation data}\label{vegetation-data}

The vegetation data describes the spatial distribution of 62 vegetation
classes in the study area. This data is in a raster format and so the
data are organized using a square grid comprising square grid cells that
are each the same size. In our case, the raster data contains multiple
layers (also called ``bands'') and each layer has corresponds to a
spatial grid with exactly the same area and has exactly the same
dimensionality (i.e.~number of rows, columns, and cells). In this
dataset, there are 62 different regular spatial grids layered on top of
each other -- with each layer corresponding to a different vegetation
class -- and each of these layers contains a grid with 343 rows, 320
columns, and 109760 cells. Within each layer, each cell corresponds to a
1 by 1 km square. The values associated with each grid cell indicate the
(one) presence or (zero) absence of a given vegetation class in the
cell.

\begin{figure}
\centering
\includegraphics{images/rasterbands.png}
\caption{}
\end{figure}

Let's explore the vegetation data.

\begin{Shaded}
\begin{Highlighting}[]
\CommentTok{# print a short summary of the data}
\KeywordTok{print}\NormalTok{(veg_data)}
\end{Highlighting}
\end{Shaded}

\begin{verbatim}
## class      : RasterStack 
## dimensions : 343, 320, 109760, 62  (nrow, ncol, ncell, nlayers)
## resolution : 1000, 1000  (x, y)
## extent     : 1080496, 1400496, -4841217, -4498217  (xmin, xmax, ymin, ymax)
## crs        : +proj=aea +lat_1=-18 +lat_2=-36 +lat_0=0 +lon_0=132 +x_0=0 +y_0=0 +ellps=GRS80 +units=m +no_defs 
## names      : vegetation.1, vegetation.2, vegetation.3, vegetation.4, vegetation.5, vegetation.6, vegetation.7, vegetation.8, vegetation.9, vegetation.10, vegetation.11, vegetation.12, vegetation.13, vegetation.14, vegetation.15, ... 
## min values :            0,            0,            0,            0,            0,            0,            0,            0,            0,             0,             0,             0,             0,             0,             0, ... 
## max values :            1,            1,            1,            1,            1,            1,            1,            1,            1,             1,             1,             1,             1,             1,             1, ...
\end{verbatim}

\begin{Shaded}
\begin{Highlighting}[]
\CommentTok{# plot a map of the 36th vegetation class}
\KeywordTok{plot}\NormalTok{(veg_data[[}\DecValTok{36}\NormalTok{]])}
\end{Highlighting}
\end{Shaded}

\begin{center}\includegraphics{prioritizr-workshop-manual_files/figure-latex/unnamed-chunk-33-1} \end{center}

\begin{Shaded}
\begin{Highlighting}[]
\CommentTok{# plot an interactive map of the 36th vegetation class}
\KeywordTok{mapview}\NormalTok{(veg_data[[}\DecValTok{36}\NormalTok{]])}
\end{Highlighting}
\end{Shaded}

\begin{Shaded}
\begin{Highlighting}[]
\CommentTok{# print number of rows in the data}
\KeywordTok{nrow}\NormalTok{(veg_data)}
\end{Highlighting}
\end{Shaded}

\begin{verbatim}
## [1] 343
\end{verbatim}

\begin{Shaded}
\begin{Highlighting}[]
\CommentTok{# print number of columns  in the data}
\KeywordTok{ncol}\NormalTok{(veg_data)}
\end{Highlighting}
\end{Shaded}

\begin{verbatim}
## [1] 320
\end{verbatim}

\begin{Shaded}
\begin{Highlighting}[]
\CommentTok{# print number of cells in the data}
\KeywordTok{ncell}\NormalTok{(veg_data)}
\end{Highlighting}
\end{Shaded}

\begin{verbatim}
## [1] 109760
\end{verbatim}

\begin{Shaded}
\begin{Highlighting}[]
\CommentTok{# print number of layers in the data}
\KeywordTok{nlayers}\NormalTok{(veg_data)}
\end{Highlighting}
\end{Shaded}

\begin{verbatim}
## [1] 62
\end{verbatim}

\begin{Shaded}
\begin{Highlighting}[]
\CommentTok{# print  resolution on the x-axis}
\KeywordTok{xres}\NormalTok{(veg_data)}
\end{Highlighting}
\end{Shaded}

\begin{verbatim}
## [1] 1000
\end{verbatim}

\begin{Shaded}
\begin{Highlighting}[]
\CommentTok{# print resolution on the y-axis}
\KeywordTok{yres}\NormalTok{(veg_data)}
\end{Highlighting}
\end{Shaded}

\begin{verbatim}
## [1] 1000
\end{verbatim}

\begin{Shaded}
\begin{Highlighting}[]
\CommentTok{# print spatial extent of the grid, i.e. coordinates for corners}
\KeywordTok{extent}\NormalTok{(veg_data)}
\end{Highlighting}
\end{Shaded}

\begin{verbatim}
## class      : Extent 
## xmin       : 1080496 
## xmax       : 1400496 
## ymin       : -4841217 
## ymax       : -4498217
\end{verbatim}

\begin{Shaded}
\begin{Highlighting}[]
\CommentTok{# print the coordinate reference system}
\KeywordTok{print}\NormalTok{(veg_data}\OperatorTok{@}\NormalTok{crs)}
\end{Highlighting}
\end{Shaded}

\begin{verbatim}
## CRS arguments:
##  +proj=aea +lat_1=-18 +lat_2=-36 +lat_0=0 +lon_0=132 +x_0=0 +y_0=0
## +ellps=GRS80 +units=m +no_defs
\end{verbatim}

\begin{Shaded}
\begin{Highlighting}[]
\CommentTok{# print a summary of the first layer in the stack}
\KeywordTok{print}\NormalTok{(veg_data[[}\DecValTok{1}\NormalTok{]])}
\end{Highlighting}
\end{Shaded}

\begin{verbatim}
## class      : RasterLayer 
## band       : 1  (of  62  bands)
## dimensions : 343, 320, 109760  (nrow, ncol, ncell)
## resolution : 1000, 1000  (x, y)
## extent     : 1080496, 1400496, -4841217, -4498217  (xmin, xmax, ymin, ymax)
## crs        : +proj=aea +lat_1=-18 +lat_2=-36 +lat_0=0 +lon_0=132 +x_0=0 +y_0=0 +ellps=GRS80 +units=m +no_defs 
## source     : /home/travis/build/prioritizr/PacMara_workshop/data/vegetation.tif 
## names      : vegetation.1 
## values     : 0, 1  (min, max)
\end{verbatim}

\begin{Shaded}
\begin{Highlighting}[]
\CommentTok{# print the value in the 800th cell in the first layer of the stack}
\KeywordTok{print}\NormalTok{(veg_data[[}\DecValTok{1}\NormalTok{]][}\DecValTok{800}\NormalTok{])}
\end{Highlighting}
\end{Shaded}

\begin{verbatim}
##   
## 0
\end{verbatim}

\begin{Shaded}
\begin{Highlighting}[]
\CommentTok{# print the value of the cell located in the 30th row and the 60th column of}
\CommentTok{# the first layer}
\KeywordTok{print}\NormalTok{(veg_data[[}\DecValTok{1}\NormalTok{]][}\DecValTok{30}\NormalTok{, }\DecValTok{60}\NormalTok{])}
\end{Highlighting}
\end{Shaded}

\begin{verbatim}
##   
## 0
\end{verbatim}

\begin{Shaded}
\begin{Highlighting}[]
\CommentTok{# calculate the sum of all the cell values in the first layer}
\KeywordTok{cellStats}\NormalTok{(veg_data[[}\DecValTok{1}\NormalTok{]], }\StringTok{"sum"}\NormalTok{)}
\end{Highlighting}
\end{Shaded}

\begin{verbatim}
## [1] 36
\end{verbatim}

\begin{Shaded}
\begin{Highlighting}[]
\CommentTok{# calculate the maximum value of all the cell values in the first layer}
\KeywordTok{cellStats}\NormalTok{(veg_data[[}\DecValTok{1}\NormalTok{]], }\StringTok{"max"}\NormalTok{)}
\end{Highlighting}
\end{Shaded}

\begin{verbatim}
## [1] 1
\end{verbatim}

\begin{Shaded}
\begin{Highlighting}[]
\CommentTok{# calculate the minimum value of all the cell values in the first layer}
\KeywordTok{cellStats}\NormalTok{(veg_data[[}\DecValTok{1}\NormalTok{]], }\StringTok{"min"}\NormalTok{)}
\end{Highlighting}
\end{Shaded}

\begin{verbatim}
## [1] 0
\end{verbatim}

\begin{Shaded}
\begin{Highlighting}[]
\CommentTok{# calculate the mean value of all the cell values in the first layer}
\KeywordTok{cellStats}\NormalTok{(veg_data[[}\DecValTok{1}\NormalTok{]], }\StringTok{"mean"}\NormalTok{)}
\end{Highlighting}
\end{Shaded}

\begin{verbatim}
## [1] 0.0003279883
\end{verbatim}

\clearpage

\begin{Shaded}
\begin{Highlighting}[]
\CommentTok{# calculate the maximum value in each layer}
\KeywordTok{as_tibble}\NormalTok{(}\KeywordTok{data.frame}\NormalTok{(}\DataTypeTok{max =} \KeywordTok{cellStats}\NormalTok{(veg_data, }\StringTok{"max"}\NormalTok{)))}
\end{Highlighting}
\end{Shaded}

\begin{verbatim}
## # A tibble: 62 x 1
##      max
##    <dbl>
##  1     1
##  2     1
##  3     1
##  4     1
##  5     1
##  6     1
##  7     1
##  8     1
##  9     1
## 10     1
## # ... with 52 more rows
\end{verbatim}

Now, you can try and answer some questions about the vegetation data.

\BeginKnitrBlock{rmdquestion}
\begin{enumerate}
\def\labelenumi{\arabic{enumi}.}
\tightlist
\item
  What part of the study area is the 51st vegetation class found in
  (hint: make a map)?
\item
  What proportion of cells contain the 12th vegetation class?
\item
  Which vegetation class is present in the greatest number of cells?
\item
  The planning unit data and the vegetation data should have the same
  coordinate reference system. Can you check if they are the same?
\end{enumerate}
\EndKnitrBlock{rmdquestion}

\chapter{Spatial prioritizations}\label{spatial-prioritizations}

\section{Introduction}\label{introduction-2}

Here we will develop prioritizations to identify priority areas for
protected area establishment. Its worth noting that
\href{https://prioritizr.net/}{prioritizr},
\href{http://marxan.org/}{Marxan}, and
\href{https://www.helsinki.fi/en/researchgroups/digital-geography-lab/software-developed-in-cbig\#section-52992}{Zonation}
are all decision support tools. This means that they are designed to
help you make decisions---they can't make decisions for you.

\section{Starting out simple}\label{starting-out-simple}

To start things off, let's keep things simple. Let's create a
prioritization using the
\href{https://prioritizr.net/reference/add_min_set_objective.html}{minimum
set formulation of the reserve selection problem}. This formulation
means that we want a solution that will meet the targets for our
biodiversity features for minimum cost. Here, we will set 5\% targets
for each vegetation class and use the data in the \texttt{cost} column
to specify acquisition costs. Unlike Marxan, we do not have to calibrate
species penalty factors (SPFs) to ensure that our target are
met---prioritizr should always return solutions to minimum set problems
where all the targets are met. Although we strongly recommend using
\href{https://www.gurobi.com/}{Gurobi} to solve problems (with
\href{https://prioritizr.net/reference/add_gurobi_solver.html}{\texttt{add\_gurobi\_solver}}),
we will use the
\href{https://prioritizr.net/reference/add_lsymphony_solver.html}{lpsymphony
solver} in this workshop since it is easier to install. The Gurobi
solver is much faster than the lpsymphony solver
(\href{https://prioritizr.net/articles/gurobi_installation.html}{see
here for installation instructions}).

\begin{Shaded}
\begin{Highlighting}[]
\CommentTok{# print planning unit data}
\KeywordTok{print}\NormalTok{(pu_data)}
\end{Highlighting}
\end{Shaded}

\begin{verbatim}
## class       : SpatialPolygonsDataFrame 
## features    : 1130 
## extent      : 1080623, 1399989, -4840595, -4497092  (xmin, xmax, ymin, ymax)
## crs         : +proj=aea +lat_1=-18 +lat_2=-36 +lat_0=0 +lon_0=132 +x_0=0 +y_0=0 +ellps=GRS80 +units=m +no_defs 
## variables   : 5
## names       :   id,              cost, status, locked_in, locked_out 
## min values  :    1, 0.192488262910798,      0,         0,          0 
## max values  : 1130,  61.9272727272727,      2,         1,          1
\end{verbatim}

\begin{Shaded}
\begin{Highlighting}[]
\CommentTok{# make prioritization problem}
\NormalTok{p1 <-}\StringTok{ }\KeywordTok{problem}\NormalTok{(pu_data, veg_data, }\DataTypeTok{cost_column =} \StringTok{"cost"}\NormalTok{) }\OperatorTok
\StringTok{      }\KeywordTok{add_min_set_objective}\NormalTok{() }\OperatorTok
\StringTok{      }\KeywordTok{add_relative_targets}\NormalTok{(}\FloatTok{0.05}\NormalTok{) }\OperatorTok\StringTok{ }\CommentTok{# 5% representation targets}
\StringTok{      }\KeywordTok{add_binary_decisions}\NormalTok{() }\OperatorTok
\StringTok{      }\KeywordTok{add_lpsymphony_solver}\NormalTok{(}\DataTypeTok{verbose =} \OtherTok{FALSE}\NormalTok{)}

\CommentTok{# print problem}
\KeywordTok{print}\NormalTok{(p1)}
\end{Highlighting}
\end{Shaded}

\begin{verbatim}
## Conservation Problem
##   planning units: SpatialPolygonsDataFrame (1130 units)
##   cost:           min: 0.19249, max: 61.92727
##   features:       vegetation.1, vegetation.2, vegetation.3, ... (62 features)
##   objective:      Minimum set objective 
##   targets:        Relative targets [targets (min: 0.05, max: 0.05)]
##   decisions:      Binary decision 
##   constraints:    <none>
##   penalties:      <none>
##   portfolio:      default
##   solver:         Lpsymphony [first_feasible (0), gap (0.1), time_limit (-1), verbose (0)]
\end{verbatim}

\begin{Shaded}
\begin{Highlighting}[]
\CommentTok{# solve problem}
\NormalTok{s1 <-}\StringTok{ }\KeywordTok{solve}\NormalTok{(p1)}

\CommentTok{# print solution, the solution_1 column contains the solution values}
\CommentTok{# indicating if a planning unit is (1) selected or (0) not}
\KeywordTok{print}\NormalTok{(s1)}
\end{Highlighting}
\end{Shaded}

\begin{verbatim}
## class       : SpatialPolygonsDataFrame 
## features    : 1130 
## extent      : 1080623, 1399989, -4840595, -4497092  (xmin, xmax, ymin, ymax)
## crs         : +proj=aea +lat_1=-18 +lat_2=-36 +lat_0=0 +lon_0=132 +x_0=0 +y_0=0 +ellps=GRS80 +units=m +no_defs 
## variables   : 6
## names       :   id,              cost, status, locked_in, locked_out, solution_1 
## min values  :    1, 0.192488262910798,      0,         0,          0,          0 
## max values  : 1130,  61.9272727272727,      2,         1,          1,          1
\end{verbatim}

\begin{Shaded}
\begin{Highlighting}[]
\CommentTok{# calculate number of planning units selected in the prioritization}
\KeywordTok{sum}\NormalTok{(s1}\OperatorTok{$}\NormalTok{solution_}\DecValTok{1}\NormalTok{)}
\end{Highlighting}
\end{Shaded}

\begin{verbatim}
## [1] 36
\end{verbatim}

\begin{Shaded}
\begin{Highlighting}[]
\CommentTok{# calculate total cost of the prioritization}
\KeywordTok{sum}\NormalTok{(s1}\OperatorTok{$}\NormalTok{solution_}\DecValTok{1} \OperatorTok{*}\StringTok{ }\NormalTok{s1}\OperatorTok{$}\NormalTok{cost)}
\end{Highlighting}
\end{Shaded}

\begin{verbatim}
## [1] 806.2393
\end{verbatim}

\begin{Shaded}
\begin{Highlighting}[]
\CommentTok{# plot solution}
\KeywordTok{spplot}\NormalTok{(s1, }\StringTok{"solution_1"}\NormalTok{, }\DataTypeTok{col.regions =} \KeywordTok{c}\NormalTok{(}\StringTok{"white"}\NormalTok{, }\StringTok{"darkgreen"}\NormalTok{), }\DataTypeTok{main =} \StringTok{"s1"}\NormalTok{)}
\end{Highlighting}
\end{Shaded}

\begin{center}\includegraphics[width=0.65\linewidth]{prioritizr-workshop-manual_files/figure-latex/unnamed-chunk-39-1} \end{center}

Now let's examine the solution.

\BeginKnitrBlock{rmdquestion}
\begin{enumerate}
\def\labelenumi{\arabic{enumi}.}
\tightlist
\item
  How many planing units were selected in the prioritization? What
  proportion of planning units were selected in the prioritization?
\item
  Is there a pattern in the spatial distribution of the priority areas?
\item
  Can you verify that all of the targets were met in the prioritization
  (hint:
  \texttt{feature\_representation(p1,\ s1{[},\ "solution\_1"{]})})?
\end{enumerate}
\EndKnitrBlock{rmdquestion}

\section{Adding complexity}\label{adding-complexity}

Our first prioritization suffers many limitations, so let's add
additional constraints to the problem to make it more useful. First,
let's lock in planing units that are already by covered protected areas.
If some vegetation communities are already secured inside existing
protected areas, then we might not need to add as many new protected
areas to the existing protected area system to meet their targets. Since
our planning unit data (\texttt{pu\_da}) already contains this
information in the \texttt{locked\_in} column, we can use this column
name to specify which planning units should be locked in.

\begin{Shaded}
\begin{Highlighting}[]
\CommentTok{# make prioritization problem}
\NormalTok{p2 <-}\StringTok{ }\KeywordTok{problem}\NormalTok{(pu_data, veg_data, }\DataTypeTok{cost_column =} \StringTok{"cost"}\NormalTok{) }\OperatorTok
\StringTok{      }\KeywordTok{add_min_set_objective}\NormalTok{() }\OperatorTok
\StringTok{      }\KeywordTok{add_relative_targets}\NormalTok{(}\FloatTok{0.05}\NormalTok{) }\OperatorTok
\StringTok{      }\KeywordTok{add_locked_in_constraints}\NormalTok{(}\StringTok{"locked_in"}\NormalTok{) }\OperatorTok
\StringTok{      }\KeywordTok{add_binary_decisions}\NormalTok{() }\OperatorTok
\StringTok{      }\KeywordTok{add_lpsymphony_solver}\NormalTok{(}\DataTypeTok{verbose =} \OtherTok{FALSE}\NormalTok{)}

\CommentTok{# print problem}
\KeywordTok{print}\NormalTok{(p2)}
\end{Highlighting}
\end{Shaded}

\begin{verbatim}
## Conservation Problem
##   planning units: SpatialPolygonsDataFrame (1130 units)
##   cost:           min: 0.19249, max: 61.92727
##   features:       vegetation.1, vegetation.2, vegetation.3, ... (62 features)
##   objective:      Minimum set objective 
##   targets:        Relative targets [targets (min: 0.05, max: 0.05)]
##   decisions:      Binary decision 
##   constraints:    <Locked in planning units [257 locked units]>
##   penalties:      <none>
##   portfolio:      default
##   solver:         Lpsymphony [first_feasible (0), gap (0.1), time_limit (-1), verbose (0)]
\end{verbatim}

\begin{Shaded}
\begin{Highlighting}[]
\CommentTok{# solve problem}
\NormalTok{s2 <-}\StringTok{ }\KeywordTok{solve}\NormalTok{(p2)}

\CommentTok{# plot solution}
\KeywordTok{spplot}\NormalTok{(s2, }\StringTok{"solution_1"}\NormalTok{, }\DataTypeTok{col.regions =} \KeywordTok{c}\NormalTok{(}\StringTok{"white"}\NormalTok{, }\StringTok{"darkgreen"}\NormalTok{), }\DataTypeTok{main =} \StringTok{"s2"}\NormalTok{)}
\end{Highlighting}
\end{Shaded}

\begin{center}\includegraphics[width=0.65\linewidth]{prioritizr-workshop-manual_files/figure-latex/unnamed-chunk-41-1} \end{center}

Let's pretend that we talked to an expert on the vegetation communities
in our study system and they recommended that a 20\% target was needed
for each vegetation class. So, armed with this information, let's set
the targets to 20\%.

\begin{Shaded}
\begin{Highlighting}[]
\CommentTok{# make prioritization problem}
\NormalTok{p3 <-}\StringTok{ }\KeywordTok{problem}\NormalTok{(pu_data, veg_data, }\DataTypeTok{cost_column =} \StringTok{"cost"}\NormalTok{) }\OperatorTok
\StringTok{      }\KeywordTok{add_min_set_objective}\NormalTok{() }\OperatorTok
\StringTok{      }\KeywordTok{add_relative_targets}\NormalTok{(}\FloatTok{0.2}\NormalTok{) }\OperatorTok
\StringTok{      }\KeywordTok{add_locked_in_constraints}\NormalTok{(}\StringTok{"locked_in"}\NormalTok{) }\OperatorTok
\StringTok{      }\KeywordTok{add_binary_decisions}\NormalTok{() }\OperatorTok
\StringTok{      }\KeywordTok{add_lpsymphony_solver}\NormalTok{(}\DataTypeTok{verbose =} \OtherTok{FALSE}\NormalTok{)}

\CommentTok{# print problem}
\KeywordTok{print}\NormalTok{(p3)}
\end{Highlighting}
\end{Shaded}

\begin{verbatim}
## Conservation Problem
##   planning units: SpatialPolygonsDataFrame (1130 units)
##   cost:           min: 0.19249, max: 61.92727
##   features:       vegetation.1, vegetation.2, vegetation.3, ... (62 features)
##   objective:      Minimum set objective 
##   targets:        Relative targets [targets (min: 0.2, max: 0.2)]
##   decisions:      Binary decision 
##   constraints:    <Locked in planning units [257 locked units]>
##   penalties:      <none>
##   portfolio:      default
##   solver:         Lpsymphony [first_feasible (0), gap (0.1), time_limit (-1), verbose (0)]
\end{verbatim}

\begin{Shaded}
\begin{Highlighting}[]
\CommentTok{# solve problem}
\NormalTok{s3 <-}\StringTok{ }\KeywordTok{solve}\NormalTok{(p3)}

\CommentTok{# plot solution}
\KeywordTok{spplot}\NormalTok{(s3, }\StringTok{"solution_1"}\NormalTok{, }\DataTypeTok{col.regions =} \KeywordTok{c}\NormalTok{(}\StringTok{"white"}\NormalTok{, }\StringTok{"darkgreen"}\NormalTok{), }\DataTypeTok{main =} \StringTok{"s3"}\NormalTok{)}
\end{Highlighting}
\end{Shaded}

\begin{center}\includegraphics[width=0.65\linewidth]{prioritizr-workshop-manual_files/figure-latex/unnamed-chunk-42-1} \end{center}

Next, let's lock out highly degraded areas. Similar to before, this data
is present in our planning unit data so we can use the
\texttt{locked\_out} column name to achieve this.

\begin{Shaded}
\begin{Highlighting}[]
\CommentTok{# make prioritization problem}
\NormalTok{p4 <-}\StringTok{ }\KeywordTok{problem}\NormalTok{(pu_data, veg_data, }\DataTypeTok{cost_column =} \StringTok{"cost"}\NormalTok{) }\OperatorTok
\StringTok{      }\KeywordTok{add_min_set_objective}\NormalTok{() }\OperatorTok
\StringTok{      }\KeywordTok{add_relative_targets}\NormalTok{(}\FloatTok{0.2}\NormalTok{) }\OperatorTok
\StringTok{      }\KeywordTok{add_locked_in_constraints}\NormalTok{(}\StringTok{"locked_in"}\NormalTok{) }\OperatorTok
\StringTok{      }\KeywordTok{add_locked_out_constraints}\NormalTok{(}\StringTok{"locked_out"}\NormalTok{) }\OperatorTok
\StringTok{      }\KeywordTok{add_binary_decisions}\NormalTok{() }\OperatorTok
\StringTok{      }\KeywordTok{add_lpsymphony_solver}\NormalTok{(}\DataTypeTok{verbose =} \OtherTok{FALSE}\NormalTok{)}
\end{Highlighting}
\end{Shaded}

\begin{Shaded}
\begin{Highlighting}[]
\CommentTok{# print problem}
\KeywordTok{print}\NormalTok{(p4)}
\end{Highlighting}
\end{Shaded}

\begin{verbatim}
## Conservation Problem
##   planning units: SpatialPolygonsDataFrame (1130 units)
##   cost:           min: 0.19249, max: 61.92727
##   features:       vegetation.1, vegetation.2, vegetation.3, ... (62 features)
##   objective:      Minimum set objective 
##   targets:        Relative targets [targets (min: 0.2, max: 0.2)]
##   decisions:      Binary decision 
##   constraints:    <Locked in planning units [257 locked units]
##                    Locked out planning units [132 locked units]>
##   penalties:      <none>
##   portfolio:      default
##   solver:         Lpsymphony [first_feasible (0), gap (0.1), time_limit (-1), verbose (0)]
\end{verbatim}

\begin{Shaded}
\begin{Highlighting}[]
\CommentTok{# solve problem}
\NormalTok{s4 <-}\StringTok{ }\KeywordTok{solve}\NormalTok{(p4)}

\CommentTok{# plot solution}
\KeywordTok{spplot}\NormalTok{(s4, }\StringTok{"solution_1"}\NormalTok{, }\DataTypeTok{col.regions =} \KeywordTok{c}\NormalTok{(}\StringTok{"white"}\NormalTok{, }\StringTok{"darkgreen"}\NormalTok{), }\DataTypeTok{main =} \StringTok{"s4"}\NormalTok{)}
\end{Highlighting}
\end{Shaded}

\begin{center}\includegraphics[width=0.65\linewidth]{prioritizr-workshop-manual_files/figure-latex/unnamed-chunk-44-1} \end{center}

\clearpage

Now, let's compare the solutions.

\BeginKnitrBlock{rmdquestion}
\begin{enumerate}
\def\labelenumi{\arabic{enumi}.}
\tightlist
\item
  What is the cost of the planning units selected in \texttt{s2},
  \texttt{s3}, and \texttt{s4}?
\item
  How many planning units are in \texttt{s2}, \texttt{s3}, and
  \texttt{s4}?
\item
  Do the solutions with more planning units have a greater cost? Why or
  why not?
\item
  Why does the first solution (\texttt{s1}) cost less than the second
  solution with protected areas locked into the solution (\texttt{s2})?
\item
  Why does the third solution (\texttt{s3}) cost less than the fourth
  solution solution with highly degraded areas locked out (\texttt{s4})?
\item
  Since planning units covered by existing protected areas have already
  been purchased, what is the cost for expanding the protected area
  system based on on the fourth prioritization (\texttt{s4}) (hint:
  total cost minus the cost of locked in planning units)?
\item
  What happens if you specify targets that exceed the total amount of
  vegetation in the study area and try to solve the problem? You can do
  this by modifying the code to make \texttt{p4} with
  \texttt{add\_absolute\_targets(1000)} instead of
  \texttt{add\_relative\_targets(0.2)} and generating a new solution.
\end{enumerate}
\EndKnitrBlock{rmdquestion}

\section{Penalizing fragmentation}\label{penalizing-fragmentation}

Plans for protected area systems should facilitate gene flow and
dispersal between individual reserves in the system. However, the
prioritizations we have made so far have been highly fragmented. Similar
to the Marxan decision support tool, we can add penalties to our
conservation planning problem to penalize fragmentation (i.e.~total
exposed boundary length) and we also need to set a useful penalty value
when adding such penalties (akin to Marxan's boundary length multiplier
value; BLM). If we set our penalty value too low, then we will end up
with a solution that is identical to the solution with no added
penalties. If we set our penalty value too high, then prioritizr will
take a long time to solve the problem and we will end up with a solution
that contains lots of extra planning units that are not needed (since
the penalty value is so high that minimizing fragmentation is more
important than cost). As a rule of thumb, we generally want penalty
values between 0.00001 and 0.01 but finding a useful penalty value
requires calibration. The ``correct'' penalty value depends on the size
of the planning units, the main objective values (e.g.~cost values), and
the effect of fragmentation on biodiversity persistence. Let's create a
new problem that is similar to our previous problem
(\texttt{p4})---except that it contains boundary length penalties and a
slightly higher optimality gap to reduce runtime (default is 0.1)---and
solve it. Since our planning unit data is in a spatial format
(i.e.~vector or raster data), prioritizr can automatically calculate the
boundary data for us.

\clearpage

\begin{Shaded}
\begin{Highlighting}[]
\CommentTok{# make prioritization problem}
\NormalTok{p5 <-}\StringTok{ }\KeywordTok{problem}\NormalTok{(pu_data, veg_data, }\DataTypeTok{cost_column =} \StringTok{"cost"}\NormalTok{) }\OperatorTok
\StringTok{      }\KeywordTok{add_min_set_objective}\NormalTok{() }\OperatorTok
\StringTok{      }\KeywordTok{add_boundary_penalties}\NormalTok{(}\DataTypeTok{penalty =} \FloatTok{0.0005}\NormalTok{) }\OperatorTok
\StringTok{      }\KeywordTok{add_relative_targets}\NormalTok{(}\FloatTok{0.2}\NormalTok{) }\OperatorTok
\StringTok{      }\KeywordTok{add_locked_in_constraints}\NormalTok{(}\StringTok{"locked_in"}\NormalTok{) }\OperatorTok
\StringTok{      }\KeywordTok{add_locked_out_constraints}\NormalTok{(}\StringTok{"locked_out"}\NormalTok{) }\OperatorTok
\StringTok{      }\KeywordTok{add_binary_decisions}\NormalTok{() }\OperatorTok
\StringTok{      }\KeywordTok{add_lpsymphony_solver}\NormalTok{(}\DataTypeTok{verbose =} \OtherTok{FALSE}\NormalTok{, }\DataTypeTok{gap =} \DecValTok{1}\NormalTok{)}

\CommentTok{# print problem}
\KeywordTok{print}\NormalTok{(p5)}
\end{Highlighting}
\end{Shaded}

\begin{verbatim}
## Conservation Problem
##   planning units: SpatialPolygonsDataFrame (1130 units)
##   cost:           min: 0.19249, max: 61.92727
##   features:       vegetation.1, vegetation.2, vegetation.3, ... (62 features)
##   objective:      Minimum set objective 
##   targets:        Relative targets [targets (min: 0.2, max: 0.2)]
##   decisions:      Binary decision 
##   constraints:    <Locked in planning units [257 locked units]
##                    Locked out planning units [132 locked units]>
##   penalties:      <Boundary penalties [edge factor (min: 0.5, max: 0.5), penalty (5e-04), zones]>
##   portfolio:      default
##   solver:         Lpsymphony [first_feasible (0), gap (1), time_limit (-1), verbose (0)]
\end{verbatim}

\begin{Shaded}
\begin{Highlighting}[]
\CommentTok{# solve problem,}
\CommentTok{# note this will take around 30 seconds}
\NormalTok{s5 <-}\StringTok{ }\KeywordTok{solve}\NormalTok{(p5)}

\CommentTok{# print solution}
\KeywordTok{print}\NormalTok{(s5)}
\end{Highlighting}
\end{Shaded}

\begin{verbatim}
## class       : SpatialPolygonsDataFrame 
## features    : 1130 
## extent      : 1080623, 1399989, -4840595, -4497092  (xmin, xmax, ymin, ymax)
## crs         : +proj=aea +lat_1=-18 +lat_2=-36 +lat_0=0 +lon_0=132 +x_0=0 +y_0=0 +ellps=GRS80 +units=m +no_defs 
## variables   : 6
## names       :   id,              cost, status, locked_in, locked_out, solution_1 
## min values  :    1, 0.192488262910798,      0,         0,          0,          0 
## max values  : 1130,  61.9272727272727,      2,         1,          1,          1
\end{verbatim}

\begin{Shaded}
\begin{Highlighting}[]
\CommentTok{# plot solution}
\KeywordTok{spplot}\NormalTok{(s5, }\StringTok{"solution_1"}\NormalTok{, }\DataTypeTok{col.regions =} \KeywordTok{c}\NormalTok{(}\StringTok{"white"}\NormalTok{, }\StringTok{"darkgreen"}\NormalTok{), }\DataTypeTok{main =} \StringTok{"s5"}\NormalTok{)}
\end{Highlighting}
\end{Shaded}

\begin{center}\includegraphics[width=0.65\linewidth]{prioritizr-workshop-manual_files/figure-latex/unnamed-chunk-47-1} \end{center}

Now let's compare the solutions to the problems with (\texttt{s5}) and
without (\texttt{s4}) the boundary length penalties.

\BeginKnitrBlock{rmdquestion}
\begin{enumerate}
\def\labelenumi{\arabic{enumi}.}
\tightlist
\item
  What is the cost the fourth (\texttt{s4}) and fifth (\texttt{s5})
  solutions? Why does the fifth solution (\texttt{s5}) cost more than
  the fourth (\texttt{s4}) solution?
\item
  Try setting the penalty value to 0.000000001 (i.e. \texttt{1e-9})
  instead of 0.0005. What is the cost of the solution now? Is it
  different from the fourth solution (\texttt{s4}) (hint: try plotting
  the solutions to visualize them)? Is this is a useful penalty value?
  Why?
\item
  Try setting the penalty value to 0.5. What is the cost of the solution
  now? Is it different from the fourth solution (\texttt{s4}) (hint: try
  plotting the solutions to visualize them)? Is this a useful penalty
  value? Why?
\end{enumerate}
\EndKnitrBlock{rmdquestion}

\clearpage

\section{Budget limited
prioritizations}\label{budget-limited-prioritizations}

In the real-world, the funding available for conservation is often very
limited. As a consequence, decision makers often need prioritizations
where the total cost of priority areas does not exceed a budget. In our
fourth prioritization (\texttt{s4}), we found that we would need to
spend an additional \$904 million AUD to ensure that each vegetation
community is adequately represented in the protected area system. But
what if the funds available for establishing new protected areas were
limited to \$100 million AUD? In this case, we need a ``budget limited
prioritization''. Budget limited prioritizations aim to maximize some
measure of conservation benefit subject to a budget (e.g.
\href{https://prioritizr.net/reference/add_max_cover_objective.html}{number
of species with at least one occurrence in the protected area system},
or
\href{https://prioritizr.net/reference/add_max_phylo_div_objective.html}{phylogenetic
diversity}). Let's create a prioritization by maximizing the number of
adequately represented features whilst keeping within a pre-specified
budget.

\begin{Shaded}
\begin{Highlighting}[]
\CommentTok{# funds for additional land acquisition (same units as cost data)}
\NormalTok{funds <-}\StringTok{ }\DecValTok{100}

\CommentTok{# calculate the total budget for the prioritization}
\NormalTok{budget <-}\StringTok{ }\NormalTok{funds }\OperatorTok{+}\StringTok{ }\KeywordTok{sum}\NormalTok{(s4}\OperatorTok{$}\NormalTok{cost }\OperatorTok{*}\StringTok{ }\NormalTok{s4}\OperatorTok{$}\NormalTok{locked_in)}
\KeywordTok{print}\NormalTok{(budget)}
\end{Highlighting}
\end{Shaded}

\begin{verbatim}
## [1] 8575.56
\end{verbatim}

\begin{Shaded}
\begin{Highlighting}[]
\CommentTok{# make prioritization problem}
\NormalTok{p6 <-}\StringTok{ }\KeywordTok{problem}\NormalTok{(pu_data, veg_data, }\DataTypeTok{cost_column =} \StringTok{"cost"}\NormalTok{) }\OperatorTok
\StringTok{      }\KeywordTok{add_max_features_objective}\NormalTok{(budget) }\OperatorTok
\StringTok{      }\KeywordTok{add_relative_targets}\NormalTok{(}\FloatTok{0.2}\NormalTok{) }\OperatorTok
\StringTok{      }\KeywordTok{add_locked_in_constraints}\NormalTok{(}\StringTok{"locked_in"}\NormalTok{) }\OperatorTok
\StringTok{      }\KeywordTok{add_locked_out_constraints}\NormalTok{(}\StringTok{"locked_out"}\NormalTok{) }\OperatorTok
\StringTok{      }\KeywordTok{add_binary_decisions}\NormalTok{() }\OperatorTok
\StringTok{      }\KeywordTok{add_lpsymphony_solver}\NormalTok{(}\DataTypeTok{verbose =} \OtherTok{FALSE}\NormalTok{)}

\CommentTok{# print problem}
\KeywordTok{print}\NormalTok{(p6)}
\end{Highlighting}
\end{Shaded}

\begin{verbatim}
## Conservation Problem
##   planning units: SpatialPolygonsDataFrame (1130 units)
##   cost:           min: 0.19249, max: 61.92727
##   features:       vegetation.1, vegetation.2, vegetation.3, ... (62 features)
##   objective:      Maximum representation objective [budget (8575.56009869836)]
##   targets:        Relative targets [targets (min: 0.2, max: 0.2)]
##   decisions:      Binary decision 
##   constraints:    <Locked out planning units [132 locked units]
##                    Locked in planning units [257 locked units]>
##   penalties:      <none>
##   portfolio:      default
##   solver:         Lpsymphony [first_feasible (0), gap (0.1), time_limit (-1), verbose (0)]
\end{verbatim}

\begin{Shaded}
\begin{Highlighting}[]
\CommentTok{# solve problem}
\NormalTok{s6 <-}\StringTok{ }\KeywordTok{solve}\NormalTok{(p6)}

\CommentTok{# plot solution}
\KeywordTok{spplot}\NormalTok{(s6, }\StringTok{"solution_1"}\NormalTok{, }\DataTypeTok{col.regions =} \KeywordTok{c}\NormalTok{(}\StringTok{"white"}\NormalTok{, }\StringTok{"darkgreen"}\NormalTok{), }\DataTypeTok{main =} \StringTok{"s6"}\NormalTok{)}
\end{Highlighting}
\end{Shaded}

\begin{center}\includegraphics[width=0.65\linewidth]{prioritizr-workshop-manual_files/figure-latex/budget-1} \end{center}

\begin{Shaded}
\begin{Highlighting}[]
\CommentTok{# calculate feature representation}
\NormalTok{r6 <-}\StringTok{ }\KeywordTok{feature_representation}\NormalTok{(p6, s6[, }\StringTok{"solution_1"}\NormalTok{])}

\CommentTok{# calculate number of features with targets met}
\KeywordTok{sum}\NormalTok{(r6}\OperatorTok{$}\NormalTok{relative_held }\OperatorTok{>=}\StringTok{ }\FloatTok{0.2}\NormalTok{, }\DataTypeTok{na.rm =} \OtherTok{TRUE}\NormalTok{)}
\end{Highlighting}
\end{Shaded}

\begin{verbatim}
## [1] 28
\end{verbatim}

\begin{Shaded}
\begin{Highlighting}[]
\CommentTok{# find out which features have their targets met when we add weights,}
\CommentTok{# note that NA is for vegetation.61}
\KeywordTok{print}\NormalTok{(r6}\OperatorTok{$}\NormalTok{feature[r6}\OperatorTok{$}\NormalTok{relative_held }\OperatorTok{>=}\StringTok{ }\FloatTok{0.2}\NormalTok{])}
\end{Highlighting}
\end{Shaded}

\begin{verbatim}
##  [1] "vegetation.1"  "vegetation.2"  "vegetation.3"  "vegetation.4" 
##  [5] "vegetation.5"  "vegetation.6"  "vegetation.7"  "vegetation.8" 
##  [9] "vegetation.11" "vegetation.12" "vegetation.13" "vegetation.14"
## [13] "vegetation.15" "vegetation.17" "vegetation.25" "vegetation.28"
## [17] "vegetation.29" "vegetation.30" "vegetation.32" "vegetation.33"
## [21] "vegetation.34" "vegetation.35" "vegetation.36" "vegetation.37"
## [25] "vegetation.38" "vegetation.39" "vegetation.40" "vegetation.45"
## [29] NA
\end{verbatim}

We can also add weights to specify that it is more important to meet the
targets for certain features and less important for other features. A
common approach for weighting features is to assign a greater importance
to features with smaller spatial distributions. The rationale behind
this weighting method is that features with smaller spatial
distributions are at greater risk of extinction. So, let's calculate
some weights for our vegetation communities and see how weighting the
features changes our prioritization.

\begin{Shaded}
\begin{Highlighting}[]
\CommentTok{# calculate weights as the log inverse number of grid cells that each vegetation}
\CommentTok{# class occupies, rescaled between 1 and 100}
\NormalTok{wts <-}\StringTok{ }\DecValTok{1} \OperatorTok{/}\StringTok{ }\KeywordTok{cellStats}\NormalTok{(veg_data, }\StringTok{"sum"}\NormalTok{)}
\NormalTok{wts <-}\StringTok{ }\KeywordTok{rescale}\NormalTok{(wts, }\DataTypeTok{to =} \KeywordTok{c}\NormalTok{(}\DecValTok{1}\NormalTok{, }\DecValTok{10}\NormalTok{))}

\CommentTok{# print the name of the feature with smallest weight}
\KeywordTok{names}\NormalTok{(veg_data)[}\KeywordTok{which.min}\NormalTok{(wts)]}
\end{Highlighting}
\end{Shaded}

\begin{verbatim}
## [1] "vegetation.20"
\end{verbatim}

\begin{Shaded}
\begin{Highlighting}[]
\CommentTok{# print the name of the feature with greatest weight}
\KeywordTok{names}\NormalTok{(veg_data)[}\KeywordTok{which.max}\NormalTok{(wts)]}
\end{Highlighting}
\end{Shaded}

\begin{verbatim}
## [1] "vegetation.52"
\end{verbatim}

\begin{Shaded}
\begin{Highlighting}[]
\CommentTok{# plot histogram of weights}
\KeywordTok{hist}\NormalTok{(wts, }\DataTypeTok{main =} \StringTok{"feature weights"}\NormalTok{)}
\end{Highlighting}
\end{Shaded}

\begin{center}\includegraphics[width=0.65\linewidth]{prioritizr-workshop-manual_files/figure-latex/budget-weights-1} \end{center}

\begin{Shaded}
\begin{Highlighting}[]
\CommentTok{# make prioritization problem with weights}
\NormalTok{p7 <-}\StringTok{ }\KeywordTok{problem}\NormalTok{(pu_data, veg_data, }\DataTypeTok{cost_column =} \StringTok{"cost"}\NormalTok{) }\OperatorTok
\StringTok{      }\KeywordTok{add_max_features_objective}\NormalTok{(budget) }\OperatorTok
\StringTok{      }\KeywordTok{add_relative_targets}\NormalTok{(}\FloatTok{0.2}\NormalTok{) }\OperatorTok
\StringTok{      }\KeywordTok{add_feature_weights}\NormalTok{(wts) }\OperatorTok
\StringTok{      }\KeywordTok{add_locked_in_constraints}\NormalTok{(}\StringTok{"locked_in"}\NormalTok{) }\OperatorTok
\StringTok{      }\KeywordTok{add_locked_out_constraints}\NormalTok{(}\StringTok{"locked_out"}\NormalTok{) }\OperatorTok
\StringTok{      }\KeywordTok{add_binary_decisions}\NormalTok{() }\OperatorTok
\StringTok{      }\KeywordTok{add_lpsymphony_solver}\NormalTok{(}\DataTypeTok{verbose =} \OtherTok{FALSE}\NormalTok{)}

\CommentTok{# print problem}
\KeywordTok{print}\NormalTok{(p7)}
\end{Highlighting}
\end{Shaded}

\begin{verbatim}
## Conservation Problem
##   planning units: SpatialPolygonsDataFrame (1130 units)
##   cost:           min: 0.19249, max: 61.92727
##   features:       vegetation.1, vegetation.2, vegetation.3, ... (62 features)
##   objective:      Maximum representation objective [budget (8575.56009869836)]
##   targets:        Relative targets [targets (min: 0.2, max: 0.2)]
##   decisions:      Binary decision 
##   constraints:    <Locked in planning units [257 locked units]
##                    Locked out planning units [132 locked units]>
##   penalties:      <Feature weights [weights (min: 1, max: 10)]>
##   portfolio:      default
##   solver:         Lpsymphony [first_feasible (0), gap (0.1), time_limit (-1), verbose (0)]
\end{verbatim}

\begin{Shaded}
\begin{Highlighting}[]
\CommentTok{# solve problem}
\NormalTok{s7 <-}\StringTok{ }\KeywordTok{solve}\NormalTok{(p7)}

\CommentTok{# plot solution}
\KeywordTok{spplot}\NormalTok{(s7, }\StringTok{"solution_1"}\NormalTok{, }\DataTypeTok{col.regions =} \KeywordTok{c}\NormalTok{(}\StringTok{"white"}\NormalTok{, }\StringTok{"darkgreen"}\NormalTok{), }\DataTypeTok{main =} \StringTok{"s7"}\NormalTok{)}
\end{Highlighting}
\end{Shaded}

\begin{center}\includegraphics[width=0.65\linewidth]{prioritizr-workshop-manual_files/figure-latex/budget-weights-2} \end{center}

\begin{Shaded}
\begin{Highlighting}[]
\CommentTok{# calculate feature representation}
\NormalTok{r7 <-}\StringTok{ }\KeywordTok{feature_representation}\NormalTok{(p7, s7[, }\StringTok{"solution_1"}\NormalTok{])}

\CommentTok{# calculate number of features with targets met}
\KeywordTok{sum}\NormalTok{(r7}\OperatorTok{$}\NormalTok{relative_held }\OperatorTok{>=}\StringTok{ }\FloatTok{0.2}\NormalTok{, }\DataTypeTok{na.rm =} \OtherTok{TRUE}\NormalTok{)}
\end{Highlighting}
\end{Shaded}

\begin{verbatim}
## [1] 26
\end{verbatim}

\begin{Shaded}
\begin{Highlighting}[]
\CommentTok{# find out which features have their targets met when we add weights,}
\CommentTok{# note that NA is for vegetation.61}
\KeywordTok{print}\NormalTok{(r7}\OperatorTok{$}\NormalTok{feature[r7}\OperatorTok{$}\NormalTok{relative_held }\OperatorTok{>=}\StringTok{ }\FloatTok{0.2}\NormalTok{])}
\end{Highlighting}
\end{Shaded}

\begin{verbatim}
##  [1] "vegetation.1"  "vegetation.2"  "vegetation.4"  "vegetation.5" 
##  [5] "vegetation.6"  "vegetation.8"  "vegetation.11" "vegetation.28"
##  [9] "vegetation.29" "vegetation.30" "vegetation.32" "vegetation.33"
## [13] "vegetation.34" "vegetation.35" "vegetation.36" "vegetation.37"
## [17] "vegetation.38" "vegetation.39" "vegetation.40" "vegetation.45"
## [21] "vegetation.49" "vegetation.50" "vegetation.52" "vegetation.53"
## [25] "vegetation.54" "vegetation.55" NA
\end{verbatim}

\BeginKnitrBlock{rmdquestion}
\begin{enumerate}
\def\labelenumi{\arabic{enumi}.}
\tightlist
\item
  What is the name of the feature with the smallest weight?
\item
  What is the cost of the sixth (\texttt{s6}) and seventh (\texttt{s7})
  solutions?
\item
  Does there seem to be a big difference in which planning units were
  selected in the sixth (\texttt{s6}) and seventh (\texttt{s7})
  solutions?
\item
  Is there a difference between which features are adequately
  represented in the sixth (\texttt{s6}) and seventh (\texttt{s7})
  solutions? If so, what is the difference?
\end{enumerate}
\EndKnitrBlock{rmdquestion}

\section{Solution portfolios}\label{solution-portfolios}

In systematic conservation planning, only rarely do we have data on all
of the stakeholder preferences and biodiversity features that we are
interested in conserving. As a consequence, it is often useful to
generate a portfolio of near optimal solutions to present to decision
makers to guide the reserve selection process. Generally we would want
many solutions in our portfolio (e.g.~1000) to ensure that our portfolio
contains a range of spatially distinct solutions, but here we will
generate a portfolio containing just six near-optimal solutions so the
code doesn't take too long to run. We will also increase the optimality
gap to obtain solutions that are more suboptimal than earlier (the
default gap value is 0.1).

\begin{Shaded}
\begin{Highlighting}[]
\CommentTok{# make problem with a shuffle portfolio}
\NormalTok{p8 <-}\StringTok{ }\KeywordTok{problem}\NormalTok{(pu_data, veg_data, }\DataTypeTok{cost_column =} \StringTok{"cost"}\NormalTok{) }\OperatorTok
\StringTok{      }\KeywordTok{add_max_features_objective}\NormalTok{(budget) }\OperatorTok
\StringTok{      }\KeywordTok{add_relative_targets}\NormalTok{(}\FloatTok{0.2}\NormalTok{) }\OperatorTok
\StringTok{      }\KeywordTok{add_feature_weights}\NormalTok{(wts) }\OperatorTok
\StringTok{      }\KeywordTok{add_binary_decisions}\NormalTok{() }\OperatorTok
\StringTok{      }\KeywordTok{add_shuffle_portfolio}\NormalTok{(}\DataTypeTok{number_solutions =} \DecValTok{6}\NormalTok{,}
                            \DataTypeTok{remove_duplicates =} \OtherTok{FALSE}\NormalTok{) }\OperatorTok
\StringTok{      }\KeywordTok{add_lpsymphony_solver}\NormalTok{(}\DataTypeTok{verbose =} \OtherTok{TRUE}\NormalTok{, }\DataTypeTok{gap =} \DecValTok{10}\NormalTok{)}
\end{Highlighting}
\end{Shaded}

\clearpage

\begin{Shaded}
\begin{Highlighting}[]
\CommentTok{# print problem}
\KeywordTok{print}\NormalTok{(p8)}
\end{Highlighting}
\end{Shaded}

\begin{verbatim}
## Conservation Problem
##   planning units: SpatialPolygonsDataFrame (1130 units)
##   cost:           min: 0.19249, max: 61.92727
##   features:       vegetation.1, vegetation.2, vegetation.3, ... (62 features)
##   objective:      Maximum representation objective [budget (8575.56009869836)]
##   targets:        Relative targets [targets (min: 0.2, max: 0.2)]
##   decisions:      Binary decision 
##   constraints:    <none>
##   penalties:      <Feature weights [weights (min: 1, max: 10)]>
##   portfolio:      Shuffle portfolio [number_solutions (6), remove_duplicates (0), threads (1)]
##   solver:         Lpsymphony [first_feasible (0), gap (10), time_limit (-1), verbose (1)]
\end{verbatim}

\begin{Shaded}
\begin{Highlighting}[]
\CommentTok{# solve problem}
\CommentTok{# note that this will contain six solutions since we added a portfolio}
\NormalTok{s8 <-}\StringTok{ }\KeywordTok{solve}\NormalTok{(p8)}

\CommentTok{# print solution}
\KeywordTok{print}\NormalTok{(s8)}
\end{Highlighting}
\end{Shaded}

\begin{verbatim}
## class       : SpatialPolygonsDataFrame 
## features    : 1130 
## extent      : 1080623, 1399989, -4840595, -4497092  (xmin, xmax, ymin, ymax)
## crs         : +proj=aea +lat_1=-18 +lat_2=-36 +lat_0=0 +lon_0=132 +x_0=0 +y_0=0 +ellps=GRS80 +units=m +no_defs 
## variables   : 11
## names       :   id,              cost, status, locked_in, locked_out, solution_1, solution_2, solution_3, solution_4, solution_5, solution_6 
## min values  :    1, 0.192488262910798,      0,         0,          0,          0,          0,          0,          0,          0,          0 
## max values  : 1130,  61.9272727272727,      2,         1,          1,          1,          1,          1,          1,          1,          1
\end{verbatim}

\begin{Shaded}
\begin{Highlighting}[]
\CommentTok{# calculate the cost of the first solution}
\KeywordTok{sum}\NormalTok{(s8}\OperatorTok{$}\NormalTok{solution_}\DecValTok{1} \OperatorTok{*}\StringTok{ }\NormalTok{s8}\OperatorTok{$}\NormalTok{cost)}
\end{Highlighting}
\end{Shaded}

\begin{verbatim}
## [1] 2169.162
\end{verbatim}

\begin{Shaded}
\begin{Highlighting}[]
\CommentTok{# calculate the cost of the second solution}
\KeywordTok{sum}\NormalTok{(s8}\OperatorTok{$}\NormalTok{solution_}\DecValTok{2} \OperatorTok{*}\StringTok{ }\NormalTok{s8}\OperatorTok{$}\NormalTok{cost)}
\end{Highlighting}
\end{Shaded}

\begin{verbatim}
## [1] 2142.312
\end{verbatim}

\begin{Shaded}
\begin{Highlighting}[]
\CommentTok{# calculate the proportion of planning units with the same solution values}
\CommentTok{# in the first and second solutions}
\KeywordTok{mean}\NormalTok{(s8}\OperatorTok{$}\NormalTok{solution_}\DecValTok{1} \OperatorTok{==}\StringTok{ }\NormalTok{s8}\OperatorTok{$}\NormalTok{solution_}\DecValTok{2}\NormalTok{)}
\end{Highlighting}
\end{Shaded}

\begin{verbatim}
## [1] 0.9663717
\end{verbatim}

\begin{Shaded}
\begin{Highlighting}[]
\CommentTok{# plot first solution}
\KeywordTok{spplot}\NormalTok{(s8, }\StringTok{"solution_1"}\NormalTok{, }\DataTypeTok{col.regions =} \KeywordTok{c}\NormalTok{(}\StringTok{"white"}\NormalTok{, }\StringTok{"darkgreen"}\NormalTok{),}
       \DataTypeTok{main =} \StringTok{"s8 (solution 1)"}\NormalTok{)}
\end{Highlighting}
\end{Shaded}

\begin{center}\includegraphics[width=0.65\linewidth]{prioritizr-workshop-manual_files/figure-latex/unnamed-chunk-53-1} \end{center}

\clearpage

\begin{Shaded}
\begin{Highlighting}[]
\CommentTok{# plot all solutions}
\NormalTok{s8_plots <-}\StringTok{ }\KeywordTok{lapply}\NormalTok{(}\KeywordTok{paste0}\NormalTok{(}\StringTok{"solution_"}\NormalTok{, }\KeywordTok{seq_len}\NormalTok{(}\DecValTok{6}\NormalTok{)), }\ControlFlowTok{function}\NormalTok{(x) \{}
  \KeywordTok{spplot}\NormalTok{(s8, x, }\DataTypeTok{main =}\NormalTok{ x, }\DataTypeTok{col.regions =} \KeywordTok{c}\NormalTok{(}\StringTok{"white"}\NormalTok{, }\StringTok{"darkgreen"}\NormalTok{))}
\NormalTok{\})}
\KeywordTok{do.call}\NormalTok{(grid.arrange, }\KeywordTok{append}\NormalTok{(s8_plots, }\KeywordTok{list}\NormalTok{(}\DataTypeTok{ncol =} \DecValTok{3}\NormalTok{)))}
\end{Highlighting}
\end{Shaded}

\begin{center}\includegraphics{prioritizr-workshop-manual_files/figure-latex/unnamed-chunk-54-1} \end{center}

\BeginKnitrBlock{rmdquestion}
\begin{enumerate}
\def\labelenumi{\arabic{enumi}.}
\tightlist
\item
  What is the cost of each of the six solutions in portfolio? Are their
  costs very different?
\item
  Are the solutions in the portfolio very different?
\item
  What could we do to obtain a portfolio with more different solutions?
\end{enumerate}
\EndKnitrBlock{rmdquestion}

\chapter{Answers}\label{answers}

This chapter contains the answers to the questions presented in the
earlier chapters. The answers are provided here so you can check if your
answers are correct.

\section{Redo Marxan analysis}\label{redo-marxan-analysis}

\subsection{Base analysis on
input.dat}\label{base-analysis-on-input.dat-1}

\BeginKnitrBlock{rmdanswer}
\begin{enumerate}
\def\labelenumi{\arabic{enumi}.}
\tightlist
\item
  Subjective.
\item
  I would say, the most obvious differences are in the bottom left
  corner, where the Marxan results are rather diffuse and selection
  frequencies are low. This is actually a very good example of some
  `issues' people have described with prioritizr. Its essentially
  related to a problem not having enough feature and cost heterogeneity
  for a decision support tool such as Marxan or prioritzr to find
  reasonable solutions.
\item
  Heterogenous cost structures help. So do features that are more
  complex or have more overlap with each other.
\end{enumerate}
\EndKnitrBlock{rmdanswer}

\section{Data}\label{data}

\subsection{Planning unit data}\label{planning-unit-data-1}

\BeginKnitrBlock{rmdanswer}
\begin{enumerate}
\def\labelenumi{\arabic{enumi}.}
\tightlist
\item
  \texttt{nrow(pu\_data)}
\item
  \texttt{max(pu\_data\$cost)}
\item
  \texttt{sum(pu\_data\$locked\_in)}
\item
  \texttt{mean(pu\_data\$locked\_in)}
\item
  \texttt{sum(pu\_data\$locked\_out)}
\item
  \texttt{mean(pu\_data\$locked\_out)}
\item
  \texttt{assert\_that(min(c(pu\_data\$locked\_in,\ pu\_data\$locked\_out))\ ==\ 0)}
  \texttt{assert\_that(max(c(pu\_data\$locked\_in,\ pu\_data\$locked\_out))\ ==\ 1)}
\item
  \texttt{all(is.finite(pu\_data\$cost))}
\item
  \texttt{assert\_that(sum(duplicated(pu\_data\$id))\ ==\ 0)}
\item
  Yes, the eastern side of Tasmania is generally much cheaper than the
  western side.
\item
  Yes, most planning units covered by protected areas are located in the
  south-western side of Tasmania.
\end{enumerate}
\EndKnitrBlock{rmdanswer}

\clearpage

\subsection{Vegetation data}\label{vegetation-data-1}

\BeginKnitrBlock{rmdanswer}
\begin{enumerate}
\def\labelenumi{\arabic{enumi}.}
\tightlist
\item
  Central-north Tasmania
\item
  \texttt{cellStats(veg\_data{[}{[}12{]}{]},\ "mean")}
\item
  \texttt{names(veg\_data){[}which.max(cellStats(veg\_data,\ "sum")){]}}
\item
  Yes, they are the same.
\end{enumerate}
\EndKnitrBlock{rmdanswer}

\section{Spatial prioritizations}\label{spatial-prioritizations}

\subsection{Starting out simple}\label{starting-out-simple-1}

\BeginKnitrBlock{rmdanswer}
\begin{enumerate}
\def\labelenumi{\arabic{enumi}.}
\tightlist
\item
  \texttt{sum(s1\$solution\_1)} \newline
   \texttt{mean(s1\$solution\_1)}
\item
  Yes, the planning units are generally spread out across most of the
  study area and they are not biased towards specific areas.
\item
  \texttt{all(feature\_representation(p1,\ s1{[},\ "solution\_1"{]})\$relative\_held\ \textgreater{}=\ 0.2)}
\end{enumerate}
\EndKnitrBlock{rmdanswer}

\subsection{Adding complexity}\label{adding-complexity-1}

\BeginKnitrBlock{rmdanswer}
\begin{enumerate}
\def\labelenumi{\arabic{enumi}.}
\tightlist
\item
  \texttt{sum(s2\$cost\ *\ s2\$solution\_1)} \newline
   \texttt{sum(s3\$cost\ *\ s3\$solution\_1)} \newline
   \texttt{sum(s4\$cost\ *\ s4\$solution\_1)}
\item
  \texttt{sum(s2\$solution\_1)} \newline
   \texttt{sum(s3\$solution\_1)} \newline
   \texttt{sum(s4\$solution\_1)}
\item
  No, just because a solution a solution has more planning units does
  not mean that it will cost less.
\item
  This is because the planning units covered by existing protected areas
  have a non-zero cost and locking in these planning units introduces
  inefficiencies into the solution. This is very common in real-world
  conservation prioritizations because existing protected areas are
  often in places that do little to benefit biodiversity \citep{r3}.
\item
  This is because some of the planning units that are highly
  degraded---based on just the planning unit costs and vegetation
  data---provide cost-efficient opportunities for meeting the targets
  and excluding them from the reserve selection process means that other
  more costly planning units are needed to meet the targets.
\item
  \texttt{sum(s4\$cost\ *\ s4\$solution\_1)\ -\ sum(s4\$cost\ *\ s4\$locked\_in)}
\item
  We get an error message stating the the problem is infeasible because
  there is no valid solution---even if we selected all the planning
  units the study area we would still not meet the targets.
\end{enumerate}
\EndKnitrBlock{rmdanswer}

\subsection{Penalizing fragmentation}\label{penalizing-fragmentation-1}

\BeginKnitrBlock{rmdanswer}
\begin{enumerate}
\def\labelenumi{\arabic{enumi}.}
\tightlist
\item
  The cost of the fourth solution is
  \texttt{sum(s4\$solution\_1\ *\ s4\$cost)} and the cost of the fifth
  solution is \texttt{sum(s5\$solution\_1\ *\ s5\$cost)}. The fifth
  solution (\texttt{s5}) costs more than the fourth solution
  (\texttt{s4}) because we have added penalties to the conservation
  planning problem to indicate that we are willing to accept a slightly
  more costly solution if it means that we can reduce fragmentation.
\item
  The solution is now nearly identical to the fourth solution
  (\texttt{s4}) and so has nearly the same cost. This penalty value is
  too low and is not useful because it does not reduce the fragmentation
  in our solution.
\item
  The solution now contains a lot of extra planning units that are not
  needed to meet our targets. In fact, nearly every planning unit in the
  study is now selected. This penalty value is too high and is not
  useful.
\end{enumerate}
\EndKnitrBlock{rmdanswer}

\subsection{Budget limited
prioritizations}\label{budget-limited-prioritizations-1}

\BeginKnitrBlock{rmdanswer}
\begin{enumerate}
\def\labelenumi{\arabic{enumi}.}
\tightlist
\item
  \texttt{names(veg\_data){[}which.min(wts){]}}
\item
  \texttt{sum(s6\$cost\ *\ s6\$solution\_1)} \newline
   \texttt{sum(s7\$cost\ *\ s7\$solution\_1)}
\item
  No, the sixth (\texttt{s6}) and seventh (\texttt{s7}) solutions both
  share many of the same selected planning units and there does not
  appear to be an obvious difference in the spatial location of the
  planning units which they do not share.
\item
  Yes. Both solutions contain adequately represent these features:
  \newline
  \texttt{r6\$feature{[}r6\$relative\_held\ \textgreater{}\ 0.2\ \&\ r7\$relative\_held\ \textgreater{}\ 0.2{]}}
  \newline
  The sixth (\texttt{s6}) is adequately represents these features too:
  \newline
  \texttt{r6\$feature{[}r6\$relative\_held\ \textgreater{}\ 0.2\ \&\ !r7\$relative\_held\ \textgreater{}\ 0.2{]}}
  \newline
  The seventh (\texttt{s7}) is adequately represents these features too:
  \newline
  \texttt{r7\$feature{[}r7\$relative\_held\ \textgreater{}\ 0.2\ \&\ !r6\$relative\_held\ \textgreater{}\ 0.2{]}}
\end{enumerate}
\EndKnitrBlock{rmdanswer}

\clearpage

\subsection{Solution portfolios}\label{solution-portfolios-1}

\BeginKnitrBlock{rmdanswer}
\begin{enumerate}
\def\labelenumi{\arabic{enumi}.}
\tightlist
\item
  No the cost are very similar. \newline
   \texttt{sum(s8\$solution\_1\ *\ s8\$cost)} \newline
   \texttt{sum(s8\$solution\_2\ *\ s8\$cost)} \newline
   \texttt{sum(s8\$solution\_3\ *\ s8\$cost)} \newline
   \texttt{sum(s8\$solution\_4\ *\ s8\$cost)} \newline
   \texttt{sum(s8\$solution\_5\ *\ s8\$cost)} \newline
   \texttt{sum(s8\$solution\_6\ *\ s8\$cost)}
\item
  No the status of the planning units are very similar in the all of the
  solutions in the portfolio. \newline
  \texttt{mean((s8\$solution\_1\ ==\ s8\$solution\_2)\ \&\ (s8\$solution\_1\ ==\ s8\$solution\_3)\ \&\ (s8\$solution\_1\ ==\ s8\$solution\_4)\ \&\ (s8\$solution\_1\ ==\ s8\$solution\_5)\ \&\ (s8\$solution\_1\ ==\ s8\$solution\_6))}
\item
  We should increase the number of the solutions in the portfolio.
\end{enumerate}
\EndKnitrBlock{rmdanswer}

\chapter{Acknowledgements}\label{acknowledgements}

Many thanks to \href{https://icons8.com}{Icons8} for providing the icons
used in this manual and to Yihui Xie for developing the
\href{http://bookdown.org}{bookdown R package} that underpins this
manual. We also thank Garrett Grolemund and Hadley Wickham for creating
one of the Rstudio screenshots used in this manual that was originally a
part of their \emph{R for Data Science} book.

\chapter{Session information}\label{session-information}

\begin{Shaded}
\begin{Highlighting}[]
\CommentTok{# print session information}
\KeywordTok{sessionInfo}\NormalTok{()}
\end{Highlighting}
\end{Shaded}

\begin{verbatim}
## R version 3.6.1 (2017-01-27)
## Platform: x86_64-pc-linux-gnu (64-bit)
## Running under: Ubuntu 16.04.6 LTS
## 
## Matrix products: default
## BLAS:   /home/travis/R-bin/lib/R/lib/libRblas.so
## LAPACK: /home/travis/R-bin/lib/R/lib/libRlapack.so
## 
## locale:
##  [1] LC_CTYPE=en_US.UTF-8       LC_NUMERIC=C              
##  [3] LC_TIME=en_US.UTF-8        LC_COLLATE=en_US.UTF-8    
##  [5] LC_MONETARY=en_US.UTF-8    LC_MESSAGES=en_US.UTF-8   
##  [7] LC_PAPER=en_US.UTF-8       LC_NAME=C                 
##  [9] LC_ADDRESS=C               LC_TELEPHONE=C            
## [11] LC_MEASUREMENT=en_US.UTF-8 LC_IDENTIFICATION=C       
## 
## attached base packages:
## [1] stats     graphics  grDevices utils     datasets  methods   base     
## 
## other attached packages:
##  [1] readxl_1.3.1       data.table_1.12.6  gridExtra_2.3      assertthat_0.2.1  
##  [5] scales_1.0.0       units_0.6-5        mapview_2.7.0      rgeos_0.5-2       
##  [9] rgdal_1.4-7        prioritizr_4.1.4.2 proto_1.0.0        raster_3.0-7      
## [13] sp_1.3-2           forcats_0.4.0      stringr_1.4.0      dplyr_0.8.3       
## [17] purrr_0.3.3        readr_1.3.1        tidyr_1.0.0        tibble_2.1.3      
## [21] ggplot2_3.2.1      tidyverse_1.2.1   
## 
## loaded via a namespace (and not attached):
##  [1] nlme_3.1-140       sf_0.8-0           satellite_1.0.1    lubridate_1.7.4   
##  [5] webshot_0.5.1      httr_1.4.1         tools_3.6.1        backports_1.1.5   
##  [9] utf8_1.1.4         R6_2.4.1           KernSmooth_2.23-15 DBI_1.0.0         
## [13] lazyeval_0.2.2     colorspace_1.4-1   withr_2.1.2        tidyselect_0.2.5  
## [17] leaflet_2.0.2      compiler_3.6.1     cli_1.1.0          rvest_0.3.5       
## [21] xml2_1.2.2         bookdown_0.15.1    classInt_0.4-2     digest_0.6.22     
## [25] rmarkdown_1.17     base64enc_0.1-3    pkgconfig_2.0.3    htmltools_0.4.0   
## [29] lpsymphony_1.12.0  fastmap_1.0.1      htmlwidgets_1.5.1  rlang_0.4.1       
## [33] rstudioapi_0.10    shiny_1.4.0        generics_0.0.2     jsonlite_1.6      
## [37] crosstalk_1.0.0    Rsymphony_0.1-28   magrittr_1.5       Matrix_1.2-17     
## [41] fansi_0.4.0        Rcpp_1.0.3         munsell_0.5.0      lifecycle_0.1.0   
## [45] stringi_1.4.3      yaml_2.2.0         plyr_1.8.4         grid_3.6.1        
## [49] parallel_3.6.1     promises_1.1.0     crayon_1.3.4       lattice_0.20-38   
## [53] haven_2.2.0        hms_0.5.2          zeallot_0.1.0      knitr_1.26        
## [57] pillar_1.4.2       uuid_0.1-2         velox_0.2.0        codetools_0.2-16  
## [61] stats4_3.6.1       glue_1.3.1         evaluate_0.14      modelr_0.1.5      
## [65] png_0.1-7          vctrs_0.2.0        httpuv_1.5.2       cellranger_1.1.0  
## [69] gtable_0.3.0       xfun_0.11          mime_0.7           xtable_1.8-4      
## [73] broom_0.5.2        e1071_1.7-2        later_1.0.0        class_7.3-15      
## [77] viridisLite_0.3.0
\end{verbatim}

\chapter{References}\label{references}

\bibliography{references.bib}


\end{document}
